\documentclass[12pt]{article}%[a4paper,twoside, 11pt]{ociamthesis}
\usepackage{amsmath}
\usepackage{amsfonts}
\usepackage{graphicx}
\usepackage{biblatex}
\usepackage[margin=3cm]{geometry}
\usepackage{color}
\usepackage{xcolor}
\usepackage{biblatex}
\usepackage{tabularx}
\usepackage{setspace}
\usepackage[T1]{fontenc}
\usepackage{lmodern}
\usepackage{mathtools, amssymb, amsthm}
\usepackage{float}
\usepackage{booktabs}

\usepackage{tikz}
\usetikzlibrary{arrows}
\usepackage{scalalistings}
%\unScalaMid  % We need "|" to have standard meaning when cspm.sty is read
\usepackage[slashRename]{cspm}
\usepackage{picinpar}
\usepackage{setspace} \doublespacing

\def\X{node {$\cross$}}
\def\inCircle#1{\raisebox{.5pt}{\textcircled{\raisebox{-.9pt}{#1}}}}
\def\m{$\mathord{\mid}$}
\def\banana#1{\mathord{(\!\|} #1 \mathord{\|\!)}}
\def\tick{
  \tikz\fill[scale=0.4](0,.35) -- (.25,0) -- (0.75,.7) -- (.25,.15) -- cycle;
} 

\def\project{\mathord{\|}}

%% \def\scalasize{\footnotesize}
%% \def\scalaInlineSize{\footnotesize}

%% \def\cspmsize{\footnotesize}
%% \def\cspmInlineSize{\footnotesize}
%% \def\cspmDisplaySize{\footnotesize}
\def\CSPMMR#1{\mbox{\cspmstyle #1}}
%\def\cspmfont{\itshape}
% Macros to control whether |...| produces Scala or CSP formatting.
\def\inlineScala{\uncspMid\scalaMid}
\def\inlineCSP{\unScalaMid\cspMid}
\lstset{columns=fullflexible}
%\inlineCSP
\cspMid

\usepackage{scalalistings}
\unScalaMid
\usepackage[slashRename]{cspm}
\def\m{$\mathord{\mid}$}
\def\CSPMMR#1{\mbox{\cspmstyle #1}}
\def\inlineCSP{\uncspMid\cspMid}
\lstset{columns=fullflexible}
%\inlineCSP
\cspMid


\definecolor{dkgreen}{rgb}{0,0.6,0}
\definecolor{gray}{rgb}{0.5,0.5,0.5}
\definecolor{mauve}{rgb}{0.58,0,0.82}
%\emergencystretch=1em

%\setstretch{2}
\mathchardef\mhyphen="2D

\usepackage{url}
\usepackage{hyperref}
\hypersetup{breaklinks=true}
\urlstyle{same}

\lstset{frame=tb,
 language=Scala,
 aboveskip=3mm,
 belowskip=3mm,
 showstringspaces=false,
 columns=flexible,
 basicstyle={\small\ttfamily},
 numbers=left,
 %backgroundcolor=\color{gray!20},
 numberstyle=\tiny\color{gray},
 keywordstyle=\color{blue},
 commentstyle=\color{dkgreen},
 stringstyle=\color{mauve},
 breaklines=true,
 breakatwhitespace=true,
 tabsize=3,
 moredelim=**[is][\color{red}]{-}{-}
}



\addbibresource{references.bib}

\title{4yp}
\author{Thomas Aston}
%\college{St Catherine's College}
%\masterssubmissiontrue
%\candidateno{1054467}
%\wordcount{**********************************}
%\renewcommand{\submittedtext}{Submitted in partial completion of }
%\degree{Part B of the Final Honour Schools of Computer Science}
%\degreedate{Trinity 2023}


\begin{document}
  \maketitle
  \tableofcontents
%  \listoffigures
%  \listoftables
%  \pagebreak

\begin{itemize}
  \item Change figure linkings
  \item 
\end{itemize}

 %\section{Introduction}

%Speed Up necessity

In a world reliant on computer systems, the correctness of those systems are vital. Indeed, simple programming errors can lead to major incidents; examples of these include an automated trader losing \$460 million\cite{KnightCapital} and the inaugural Ariane 5 flight breaking up after launch due to an overflow error\cite{Flight501Failure}. 

In order to achieve better performance, concurrency can often be introduced to improve the performance programs or systems - especially those with semi-independent tasks or components.
Concurrent systems, be this on a single computer or distributed across a network, achieve these performance improvements at the cost of additional complexity as the design now needs to consider the interactions and exchange of data between threads. 
Each of different possible interactions between threads could potentially lead to a \emph{race condition} in a poorly designed systems; this is where two non-independent actions can occur in an order which produces an incorrect or unwanted outcome. 
%The most famous example of this was in the Therac-25 radiation therapy machine which lead to the deaths of three patients due to overexposure as a result of a race condition in the machine's software where the high radiation beam was selected and quickly changed to the lower radiation beam\cite{BugSnag}. 
Race conditions can be very damaging in practice - the Therac-25 radiation therapy machine killed three of its patients by radiation overexposure as a result of a race condition\cite{BugSnag}. This was caused by the operator quickly changing from the high radiation beam mode to select the lower radiation beam instead; the race condition resulted in the machine erroneously still using the high radiation beam instead. This highlights the importance of thorough system validation; had more through system validation been completed, these deaths would have likely been avoided\cite{AGift}.


%The Therac-25 radiation therapy machine had an example of this, where an operator quickly changing from the high radiation beam to select the lower radiation beam would still result in the machine still using the high radiation beam because of a race condition. This error resulted in the deaths of three patients\cite{BugSnag} and, had more through system validation been completed, these deaths would have likely been avoided\cite{AGift}.

There are two main approaches to developing correct software: testing and verification. Though thorough unit testing can be effective in minimising software bugs\cite{MicroTest}, this form of testing is significantly less effective in concurrent contexts. Testing can only establish the presence of a bug, not the absence of any; this can be somewhat addressed by writing exhaustive tests to cover every possible edge case, however this imposes severe restrictions on the complexity of such systems. Writing exhaustive tests is near impossible for sufficiently complex concurrent systems. This is predominantly due to the sheer number of different interactions between independent threads: considering all possible interactions and testing them effectively are both challenging tasks.
\emph{Linearizability} testing is an effective alternative approach, although this this relies on the random testing of edge cases; clearly this is not exhaustive either\cite{LoweLin}.

By contrast, formal verification can be used to show that systems satisfy some desired properties\cite{PrinciplesOfModelChecking}. This, however, is a complex process and it is often impossible to model check the complete behaviour of a large system simply due to the size of the resulting state space. 
We instead focus on modelling the concurrent interactions between threads; if the interactions between threads behave as expected then system validation can be reduced to checking the behaviour of the individual, sequential, threads.
Focusing on these interactions allows us to decrease the size of a system to the extent where it is now feasible to model check these behaviours. 

We choose to use the process algebra Communicating Sequential Processes (CSP) as our tool for modelling these interactions; CSP is a language for describing processes that can interact both with their environment and other parallel processes. As CSP is covered in the Part B Concurrency course\cite{Concurrency} we assume familiarity with CSP as described in \cite{Roscoe}.


%As a result, we can focus our modelling on the concurrent interactions between threads, be this directly via message passing or through concurrent datatypes or primitives. 
%If the interactions between threads behave as expected then system validation can instead be reduced to checking the behaviour of the individual sequential threads. Focussing on these interactions decreases the size of the system to the extent where it is now feasible to model check the concurrent behaviours. We choose to use the process algebra Communicating Sequential Processes (CSP)\cite{Roscoe} as our tool for modelling these interactions

\subsection{Contributions}

The contributions of this project are as follows:

\begin{itemize}
  \item We produce a CSP model of a 2-thread synchronisation object and how they can be organised into an arbitrary n-thread barrier synchronisation object. We then use the CSP model to prove the correctness of the barrier synchronisation object.
  \item We model both the standard JVM monitor and the SCL\cite{CP} implementation of a monitor, proving that the SCL monitor provides the same mutual exclusion and correctness properties as the JVM monitor but without the same unwanted behaviours.
  \item We examine a variety of locks from \cite{CADS} and a number of desirable properties of locks, using CSP refinement checks to determine which locks satisfy. We also discuss the feasibility of modelling infinite properties using CSP.
\end{itemize}



% General correctness importance
% Concurrency - Therac \cite{BugSnag}
%Types of testing
%Why formal verification
%Contributions
 %\section{Related Work}


Current literature highlights three main approaches to formal verification of software. Such verification can either be done via verification tests written during development, through automated translation of code after development or alternatively through modelling by hand.

Writing verification proofs throughout development allows developers to ensure that their code meets their specification during development.
AWS provide an example of this approach, using formal verification methods in code development in order to ensure that their code always meets its specifications \cite{Amazon1,Amazon2}. This process involve writing a specification for each function in the form of a number of pre- and post-conditions, with these properties validated automatically. Though similar in style to standard unit-tests, these actual proofs that are continually checked during development, with routine checking of these proofs indicating the correctness (or otherwise) of the code.
Though this approach requires more effort during the development and writing process, the ability to detect system design issues during implementation is significantly more helpful than via system testing at the end of development. To aid with this, some languages such as Dafny \cite{Dafny} have been developed with built in model checkers as an extension of this approach; these are however yet to reach mainstream adoption.

Automated code translation is similar, but tends to focus on more generic properties such as detecting deadlocks. NASA developed the Java PathFinder (JPF) in 1999 as one of the first examples of automated verification of code \cite{NASA1}. This was able to detect and alert on deadlocks, unhandled exceptions and assertions, but could not check for correctness against some specification \cite{NASA2}. This limits its utility significantly; code not deadlocking is require but not sufficient for correctness. More modern tools such as Stainless are able to verify some further system properties when given some additional information by the developer \cite{C4DT}. These approaches tend to be inefficient; automated translation has no inherent knowledge about restrictions on the usage of parameters or datatypes, resulting in a potentially very inefficient model.

The final approach to formal verification is to write a model of the code by hand and then verify that model. This approach is more involved and limited in scope than the two alternative methods, but can lead to a larger range of results and additional proofs that the code fulfils certain properties beyond just correctness. It also has the benefit that it is less affected by the state space explosion problem than automated tools. 

The state space explosion problem is that adding an additional process or parameter often leads to an exponential increase in the number of states generated by model checkers such as FDR or SPIN \cite{RoscoeUCS}\framebox{better description}. Indeed, a single process with a parameter $t : S$ and $k$ states  can have $k^{t}$ states; a network of $N$ of these processes can have $N^{k^{t}}$ total states. This quickly becomes infeasible to check for even relatively small values of $N$, $k$ and $t$. Additionally, a queue can have a potentially infinite number of possible permutations, leading to FDR being unable to generate the complete state space to model check. Insights and smart design can be used to create a smaller, but still correct, model than automated translation methods. This allows for model checking of system with more threads or a wider range of parameter values, leading to greater confidence in their correctness. We therefore focus on this approach.


% By contrast, verification of code by hand using some modelling language only requires verifying a final design or implementation. This can be beneficial as the greater insight gained from 

% We instead choose to focus on approach of formal verification of code by hand. This allows us to make use of our insights to optimise the models developed, resulting in a smaller model to verify and hence allowing us to verify the system for larger numbers of threads or other parameter limitations. Although 

We choose to use CSP for this task. CSP is very suitable for modelling concurrent systems with tightly restricted communications between threads \cite{Lawrence2005} and allows for natural modeling of system behaviour. Through this, we can effectively use CSP and FDR to model and then check a range of primitives and desired properties. This led to significant verification results previously, most prominently being Lowe's detection of a man-in-the-middle attack on the Needham-Schroeder protocol \cite{LoweNeedham}. It has also been used previously to model software running on the International Space Station, proving that such systems were free of deadlocks\cite{DeadlockAnalysis}. 

Indeed, CSP has also been successfully used to find bugs in concurrency primitives. Lowe used CSP to model an implementation of a concurrent channel, with FDR returning that the implementation was not deadlock free \cite{LoweDeadlock}. This bug was a very niche edge case that required a trace of 37 separate events which had yet to be spotted by hand. The produced trace allowed for a straightforward fix to the code to made to remove the deadlock; this makes it well suited to our needs here as we can both accurately model concurrent datatypes and then easily interpret any resulting error traces.

There are three main styles of concurrent programming as highlighted in \cite{CADS}: lock-based, message passing and dataype-based concurrent programming. There exists literature on formal verification of the latter two; Lowe has previously proved the correctness of a lock-free queue\cite{LFQueue} and also an implementation of a generalised alt operator \cite{LoweAlt}. There is also more general work on the verification of lock-free algorithms, such as Schellhorn and B{\"a}umler \cite{Schellhorn}. Their work uses an extended form of linear temporal logic (LTL) and the rely-guarantee paradigm (introduced by \cite{Jones}) to prove linearizability and lock-freeness properties. 

By contrast, there is an lack of research into lock-based concurrency primitives; we therefore focus on this area.\framebox{better ending}

%\emph{https://link.springer.com/chapter/10.1007/978-3-642-17511-4\_20}




 \section{Modelling and analysing implementations of locks}
\label{sec:locks}



% \inlineScala

% In this section we will analyse a number of different lock implementations. 

% The primary purpose of locks is to provide \emph{mutual exclusion} between threads; that is to avoid two threads from operating concurrently on the same section of code, referred to as the \emph{critical region}. A good lock should also fulfil some \emph{liveness} requirements, essentially that something good will eventually happen. We will present a few models of locks and examine how we can model certain liveness and safety properties using CSP.

% %When devising liveness requirements we assume that no thread wil hold the lock indefinitely; otherwise most reasonable liveness requirements can be invalidated by a thread that gains the lock and never releases it. \emph{Deadlock freedom} is a liveness requirement that if some thread is attempting to acquire the lock then some thread will eventually succeed in acquiring the lock, unless a thread holds the lock indefinitely. \emph{Starvation freedom} is a liveness requirement that any thread that tries to gain the lock will eventually succeed; by contrast deadlock freedom allows one thread to never obtain the lock as long as others complete an infinite number of critical sections. Other requirements/useful properties of locks will be explored later.

% \subsection{External interfaces}

% The most straightforward interface of a lock can be seen in Figure \ref{code:LockInterface}. This provides a |lock| function for a thread to attempt to gain the lock (blocking if some other thread currently hold the lock) and an |unlock| function for a thread to release the lock. 

% \begin{figure}[H]
% \begin{scala}
%   trait Lock{
%     /** Acquire the Lock. */
%     def lock : Unit
%     /** Release the Lock. */
%     def unlock : Unit 
%     ...
%   }
% \end{scala}
% \caption{A Scala interface for a simple lock}
% \label{code:LockInterface}
% \end{figure}

% \framebox{MOVE}
% When a thread |t| uses a lock |l| with there are four main events of importance to model in CSP:

% \begin{itemize}
%   \item |callLock.l.t| : The thread calls the lock function;
%   \item |lockAcquired.l.t| : The thread exits the lock function, now holding the lock;
%   \item |lockReleased.l.t| : The thread has called the unlock function and the unlock function has been executed to the point where a thread can now reobtain the lock
%   \item |end.t| : The thread will make no further calls to the lock; this can be used to indicate that the thread has terminated, been permanently descheduled
% \end{itemize}

% Throughout the paper, we will use |callX| to represent a thread calling function |X| and |end.t| to represent a thread terminating. The set of all threads is |ThreadID :: T.{0 .. NTHREADS - 1}|. We will now specify some ideal properties of locks using these channels:

% \subsubsection{Mutual Exclusion}\label{mutual-exclusion}
% Mutual exclusion is a safety property which states that at most one thread; i.e.~that once thread A obtains the lock, no other thread can obtain the lock until thread A unlocks. We can therefore deduce that a lock |l| with model |X| satisfies the trace refinement:  

% %This specifies that at most one thread may be in its critical section at any one time; i.e.~that once thread A obtains the lock, no other thread can obtain the lock until thread A unlocks. We can therefore deduce that a lock |l| with model |X| satisfies the trace refinement:  
  
%   \begin{cspm}
%     Mutex = lockAcquired.l?t -> lockReleased.l.t -> Mutex
%     assert Mutex [T= X \ (£$\Sigma$£ - [|lockAcquired.l, lockReleased.l|])
%   \end{cspm}

% \subsubsection{Deadlock Freedom}\label{deadlock-freedom}
% This specifies that if some thread attempts to acquire the lock then a thread will succeed in acquiring the lock\cite{TAoMP}. 

% %This does allow a CSP deadlock \framebox{Need to explain earlier} if no thread is attempting to acquire the lock, but only if the following holds: \framebox{format this}
  
%   % $\forall(s,ref) \in failures(P) \, . \,\, \#(s \downarrow callLock)= \#(s \downarrow lockAcquired) \implies ref \subset \Sigma $
  
%   % This can be captured by the following failures refinement on lock |l| with the set of all threads called |ThreadID|. This process can non-deterministically deadlock when no threads are attempting to obtain the lock and otherwise ensures that if a thread attempts to acquire the lock then some thread obtains the lock

%   We can express this in the stable-failures model by ensuring that |lockAcquired.l| is always available to be communicated when some thread has called the lock but not yet obtained it.

%   \begin{cspm}
%     AcquireLock(l, ts) = 
%          callLock.l?t:(diff(ThreadID, ts)) -> AcquireLock(l, union(ts, {t}))
%       [] lockAcquired.l?t:ts -> AcquireLock(l, diff(ts, {t}))

%     assert AcquireLock(l, {}) [F= 
%         X \ (£$\Sigma$£ - {|callLock.l, lockAcquired.l|})
%   \end{cspm}
%   |AcquireLock| takes two parameters: |l| is the identity of the lock and |ts| is the set of threads currently which have communicated a |callLock| but haven't yet acquired the lock. In our refinement check, we will assume that no lock events have occurred prior; ie. no threads have already attempted to acquire the lock. %\framebox{Explain lack of end}
%   %|AcquireLock| takes three parameters: |l| is the identity of the lock, |ts| is the set of threads currently which have communicated a |callLock| but haven't yet acquired the lock. Finally, |TS| is the set of all live threads; this is so that we can accurately restrict the threads that . 

% \subsubsection{Livelock Freedom}\label{livelock-freedom}
% %This specifies that the system must make actual progress; i.e.~that threads can't repeat actions indefinitely without making any progress. 
% This requirement specifies that the number of internal actions on a lock must be bounded while no thread holds the lock; i.e.~threads can't indefinitely repeat actions whilst the lock is unheld. This can be captured in the failures-divergences model, using a specification process parametised over lock |l| and a failures-divergences refinement against a system that only has |lockAcquired| and |lockReleased| as visible communications.

% \begin{cspm}
% LiveUnlocked(l) =   lockAcquired.l?t -> LiveLocked(l)
%                   |~| STOP
% LiveLocked(l) =     lockReleased.l?t -> LiveUnlocked
%                   |~| DIV

% assert LiveUnlocked(l) [FD= X \ (£$\Sigma$£ - [|lockAcquired.l, lockReleased.l|])
% \end{cspm}

% This specification allows the lock to diverge only when it is held by some thread and to be divergence free otherwise. This forces the number of internal actions when the lock is not held to be finite (else it could diverge and the refinement would fail), indicating that no livelock has occured. We allow the specification to non-deterministically |STOP| when the lock is unheld; this models the effective behaviour of the lock after all threads terminate.


% These are three almost essential properties of useful locks; we will consider starvation-freedom later.


% \subsection{A simple lock specification} 

% Figure \ref{code:LockSpec} shows a simple trace specification for a lock, where |l| is the identity of the lock and |ts| is the set of threads that can interact with the lock. 

% %This specification has the required property of mutual exclusion - once a thread has performed a |lockAcquired.l.t|, no other threads can acquire the lock until after the lock has been released. It also satisfies deadlock-freedom since it can always communicate a |callLock| unless either |ts == TS| (in which case some thread can communicate a |lockAcquired| then |lockReleased|) or |TS = {}| (where all threads have 'terminated' via |exit| and hence is deadlock-free since no threads will attempt to obtain the lock). Livelock-freeness is also satisfied as all actions performed make progress towards obtaining the lock or releasing the lock once it is held. 

% \begin{figure}
% \begin{cspm}
% LockSpec(l, ts) = SpecLock(l) [|{|lockAcquired.l, lockReleased.l|}|]
%                       (||| t <- ThreadID @ SpecThread(l, t))
% SpecThread(l, t) =    
%      callLock.l.t -> lockAcquired.l.t -> lockReleased.l.t -> SpecThread(l, t)
%   [] end.t -> STOP
% SpecLock(l) = lockAcquired.l?t -> lockReleased.l.t -> SpecLock(l)
% \end{cspm}
% \caption{A non-starvation-free trace specification for a lock}
% \label{code:LockSpec}
% \end{figure}

% This specification has the properties of mutual exclusion, livelock-freedom and deadlock-freedom; we have verified this by running the three assertions from \ref{mutual-exclusion}, \ref{deadlock-freedom} and \ref{livelock-freedom}. As a result, any process which failures refines\framebox{check} this specification also has these three properties. \framebox{explanation of why?}

% \subsection{Test-and-Set Lock}

% The Test-and-Set (TAS) lock implementation is based on using an |AtomicBoolean| called |state| to capture whether the lock is currently held, with |true| indicating that some thread holds the lock and |false| otherwise. The |AtomicBoolean|, has atomic |get| and |set| operations to read and write values respectively. It also has a |getAndSet| operation which atomically sets the value of the Boolean and returns the old value. The Scala code can be seen in Listing \ref{fig:TASScala}. |state| being false is equivalent to the lock being unlocked; a communication of |getAndSet(true)| with previous value |false| indicates that that thread has now obtained the lock. % that no thread previously held the lock; the thread can then release the lock by setting the value back to |false|.%When a thread attempts to obtain the lock, it performs a |state.getAndSet(true)|; a |getAndSet(true)| that returns |false| can be treated as having gained the lock, whereas a |true| indicates that some other thread already holds the lock. To release the lock a |set(false)| is done to mark the lock as available to other threads.

% \begin{figure}
%   \begin{scala}
%   import java.util.concurrent.atomic.AtomicBoolean

%   /** A lock based upon the test-and-set operation 
%     * Based on Herlihy & Shavit, Chapter 7. */
%   class TASLock extends Lock{
%     /** The state of the lock: true represents locked */
%     private val state = new AtomicBoolean(false)

%     /** Acquire the Lock */ 
%     def lock = while(state.getAndSet(true)){ }

%     /** Release the Lock */
%     def unlock = state.set(false)
%   }
%   \end{scala}
%   \caption{Test-and-set lock from \cite{CADS} \label{fig:TASScala}}
% \end{figure}

% \inlineCSP

% In order to model the TAS lock, we first need a process that acts as an |AtomicBoolean| to model the |state| variable. Figure \ref{csp:Variable} introduces a process |AtomicVar| than takes an initial value, and channels |get, set : ThreadID.T| and |getAndSet : ThreadID.T.T| for some arbitrary type |T|.

% \begin{figure}
%   \begin{cspm}
% AtomicVar(value, get, set, gAS) = 
%      get?_!value -> AtomicVar(value, get, set, gAS)
%   [] set?_?value' -> AtomicVar(value', get, set, gAS)
%   [] gAS?_!value?value' -> AtomicVar(value', get, set, gAS) 
%   \end{cspm}
%   \caption{A process encapsulating an Atomic variable with get, set and getAndSet operations}
%   \label{csp:Variable}
% \end{figure}

% We therefore represent the |state| variable from the Scala implementation by the following CSP:
% \begin{cspm}
%   channel get, set: ThreadID . Bool
%   channel getAndSet: ThreadID . Bool . Bool
%   State = Var(false, get, set, getAndSet)
%   InternalChannels = {|get, set, getAndSet|}
% \end{cspm}
% A communication on any of the channels is equivalent to a thread calling the corresponding operation in Scala. We use |false| to indicate that no thread holds the lock initially.

% We next model the operations of the lock itself. Both operations are trivial to convert, and we can linearize |Lock(t)| when the communication |getAndSet.t.False.True| occurs, indicating that |t| has obtained the lock.

% \begin{cspm}
%   Unlock(t) = setState.t!False -> SKIP -- def unlock = state.set(false)

%   Lock(t) = getAndSet.t?v!True -> if v == False then SKIP 
%                                     else Lock(t)
% \end{cspm}

% % The |Lock| procedure is also trivial, with the thread just communicating over |getAndSet|. The procedure terminates once the |getAndSet| communicates that the original value of the |state|variable was false; a |getAndSet.t.False.True| event in a trace can be linearized as the point at which the thread |t| obtains the lock.


% We model the threads that are attempting to obtain the lock by a process |Thread(x)|, where |x| is the identity of the thead. Each thread can non-deterministically chooses to either terminate or obtain the lock, release the lock and repeat. %; we use external choice here so that we can regulate the  behaviour of the threads when analysing the lock's properties. 
% Here we use |L.0| as the identity of the lock.

% \begin{cspm}
%   Thread(t) =   £$\,$£callLock.L.0.t -> Lock(t); Unlock(t); Thread(t)
%               [] end.t -> SKIP
% \end{cspm}

% Finally, we construct the lock from its components. We first synchronise all the threads over the |get|, |set| and |getAndSet| channels with the |State| process. Since |getAndSet.t.False.True| and |setState.t.False| are the linearisation points of thread |t| obtaining and releasing the lock, we can rename these communications to |lockAcquired.L.0.t| and |lockReleased.L.0.t| respectively to produce |ActualSystemR|. Finally, to obtain a system that only visibly communicates the four previously identified events, we hide the internal channels of the lock to produce |ActualSystemRExtDiv|.

% \framebox{Diagram?}

% \begin{cspm}
%   -- All initially do not hold the lock
%   AllThreads = ||| t : ThreadID @ Thread(t)
%   -- Allow all threads to peform actions on the state variable
%   ActualSystem = (AllThreads [|InternalChannels|] State)
%   -- Rename lock acquisition and releasing and hide internal events
%   ActualSystemR = ActualSystem 
%                     [[getAndSet.t.False.True <- lockAcquired.L.0.t, 
%                      £$\!\!\:\:\!\!\:\!\:$£set.t.False            £$\!\!\:\:\!\!\:\!\;$£<- lockReleased.L.0.t  | t <- ThreadID]]
%   ActualSystemRExtDiv = ActualSystemR \ InternalChannels
% \end{cspm}

% \subsubsection{Analysis}

% We firstly examine whether this model fulfils the aforementioned properties. The mutual exclusion, deadlock freedom and livelock-freedom tests from sections \ref{mutual-exclusion}, \ref{deadlock-freedom} and \ref{livelock-freedom} respectively pass. The model also does not diverge before it is first held; these are all expected results\framebox{Why?}. The TAS lock is also equivalent under traces with the earlier lock specification. However, the model can diverge whenever the lock is held. This occurs when a thread (or threads) attempting to obtain the lock a thread attempting to obtain the lock perform an infinite number of |getAndSet| operations; an example trace of this behaviour where |T.0| obtains the lock |l| follows
% \begin{cspm}
%   £$\langle$£callLock.l.T.0, callLock.l.T.1, getAndSet.T.0.False.True£$\rangle$£^
%     £$\langle$£getAndSet.T.1.True.True£$\rangle^\omega$£
% \end{cspm}
  
% This behaviour is expected; thread |T.1| is trying to obtain the lock and is being blocked by |T.0| which holds the lock. This behaviour is, however, problematic for low-level performance. Any |getAndSet| operation causes a broadcast on the shared memory bus between the processors, delaying all processors \framebox{Ref}. This also forces each thread to invalidate the value of |state| from the caches, regardless of whether the value has actually been changed. Since the above trace never results in thread |T.1| successfully setting |state|, it is preferrable to limit the number of |getAndSet| operations without unneccessarily delaying a thread from obtaining the lock. As a result, we use less costly |get| operations in order to limit the usage of |getAndSet| operations; these |getAndSet| operations are instead limited to situations where they are likely to obtain the lock. Since |get| does not change the underlying value of a variable, the read will result in at most one cache-miss per |set|/|getAndSet| on |state|; this is a marked improvement\framebox{Level of detail regarding memory buses, performance, caching etc} 


% \subsection{Test-and-Test-and-Set Lock}

% \begin{figure}
%   %\begin{Scala}
%   \begin{scala}
%     class TTASLock extends Lock{ 
%       £\dots£
%       /** Acquire the lock */
%       def lock = 
%         do{
%           while(state.get()){ } // spin until state = false
%         } while(state.getAndSet(true)) // if state = true, retry
%       £\dots£
%     }
%   \end{scala}
%   %\end{Scala}
%   \caption{Test-and-test-and-set lock from \cite{CADS}  \label{scala:TTAS}}
% \end{figure}

% The Test-and-Test-and-Set (TTAS) lock makes use of this improvement, whilst otherwise remaining similar to the TAS lock. The sole change is to the |lock| function, as can be seen in Figure \ref{scala:TTAS}, which we then specify in our CSP model as the following: 
% \begin{cspm}
%   Lock(t) =  getState.t?s -> if s = True then Lock(t)
%                              else gASState.t?v!True -> if v = False then SKIP
%                              else delay!t -> Lock(t)
% \end{cspm}
% The TTAS lock can still produce traces with threads performing an unbound number of consecutive operations, however these are now |get| operations instead of |getAndSet| operations. %Whereas the TAS lock can have an unbounded number of |getAndSet| operations for each time the lock is obtained, the 
% The TTAS lock has performs at most one |getAndSet| operation per thread when the lock becomes available. This is the case when each thread's last communication was |get.t.False|, indicating that the lock is available and hence leading to a |getAndSet| communication to attempt to gain the lock.

% We now have a linear bound on the number of unsuccessful getAndSet operations, resulting in much more efficient usage of caching and shared memory. This has been verified by synchronising with a regulator process which outputs on an error channel if |n| |getAndSet|s occur in one locking cycle; this regulator acts as a watchdog. Since the trace refinement is satisfied, we have that no |error| event has been communicated and hence at most |n| getAndSets can occur every time the lock is released. 

% \begin{cspm}
% channel error
% Reg(x) =    gASState?_ -> if (x < card(ThreadID)) then Reg(x+1)
%                           else error -> STOP
%          [] lockReleased?_ -> Reg(0)

% assert LockSpec(L.0, {}, ThreadID) 
%          [T= ActualSystemR [|{|gASState, lockReleased|}|] Reg(0)
% \end{cspm}

% \subsection{Peterson lock}

% The Peterson lock is 



% \subsection{Tree lock}

% \framebox{Might be worth introducing Peterson lock for starvation freedom both earlier and here?}

% Suppose we have an implementation of a lock that works for upto n threads and now wanted to extend this to work with more threads trying to obtain a single lock. One approach to solving this problem is to arrange a number of the n thread locks into a tree structure. The threads are assigned a leaf node and, once they have obtained that lock, they progress up the tree obtaining the next lock and so on. Once the thread reaches the root of the tree and has obtained the 'root lock' and hence holds the lock; to unlock, the thread unlocks the root lock and progressively unlocks all the locks it held until it is back at the leaf lock. To consider a simple case where n = 2 and NThreads = 8 we have the following structure: \framebox{Draw properly}

% \begin{figure}
% \begin{verbatim}
%                                   L.0
%                                 /     \                    
%                                /       \
%                               /         \
%                              /           \
%                            L.1           L.2
%                           /  \           /  \
%                          /    \         /    \
%                         /      \       /      \
%                        L.3     L.4    L.5     L.6
%                       /  |     / |    |  \    |  \
%                      /   |    /  |    |   \   |   \
%                  T.3.1   | T.4.1 |    | T.5.2 |   T.6.2
%                       T.3.2   T.4.2    T.5.1   T.6.1
% \end{verbatim}
% \caption{An example of the tree lock with 2 threads per lock and 8 threads \label{fig:Tree}}
% \end{figure}

% Here, for T.5.1 to obtain the root lock L.0, it must first obtain L.5 then L.2 then it can attempt to lock L.0. If it holds L.0 then all of the other 7 threads can't enter their critical section. Once T.5.1 wants to release the lock it unlocks L.0, then unlocks L.2 then finally unlocks L.5. T.5.1 can now either terminate or try to reobtain the root lock.

% \subsubsection{Modelling}

% We model the tree structure so that the structure of the tree is independent from the implementation of the individual locks; this will allow us to examine how different the different properties of the locks affect the overall structure.

% For convenience, we shall call the threads |T.{y}.{z}| where y is the leaf lock that the thread will first try and obtain and z is just an index over the threads starting at that leaf lock. Since the thread should be agnostic to the implementation of the lock, the thread initially either exits or calls |LockTree| to try and obtain the lock. Once the thread has obtained the lock, it then releases it by a call to |UnlockTree|. Both |LockTree| and |UnlockTree| are recursively defined, traversing their way up and down the tree respectively before terminating once they have locked the root lock or unlocked a leaf lock respectively. 
% |nextUnlock(x, y)| is used as a helper function to generate the next lock to release 

% \begin{cspm}
%   nextUnlock(x, y) = if (y / 2 > x) then nextUnlock(x, y/2) else L.y 

%   LockTree(t, L.0) = SKIP
%   LockTree(T.y.z, L.x) = Lock(T.y.z, L.x); LockTree(T.x.y, L.((x-1)/2))

%   UnlockTree(T.y.z, L.x) = if (x == y) then Unlock(T.y.z, L.x); SKIP
%                            else Unlock(T.y.z, L.y); 
%                                 UnlockTree(T.y.z, nextUnlock(x, y))

%   Thread(T.y.z) = callLock.L.y.T.y.z -> LockTree(T.y.z, L.y); 
%                                           UnlockTree(T.y.z, L.0); 
%                                           Thread(T.y.z)
%                  [] end.T.y.z -> SKIP
% \end{cspm}

% We can use the |lockAcquired| and |lockReleased| events of the root |L.0| lock as the points where a given thread obtains and releases the lock and the |callLock| events of the leaf nodes to represent when a thread tries to access the lock. To check that this lock fulfils the required properties we will hide all internal locking events; we do not care how the lock functions internally as long as it follows the required specifications. We will also rename the the three above events to appear to occur on the lock L.0. The system is therefore constructed as follows:

% \begin{cspm}
%   -- Initialise threads to not hold their locks
%   AllThreads = ||| (T.y.z) : ThreadID @ Thread(T.y.z)
%   -- Initialise all the locks
%   AllLocks = ||| (L.x) : LockID @ LockSystem(L.x)
%   -- Sychronise the threads with all the locks
%   TreeInternal = (AllThreads [|LockEvents|] AllLocks)
%   -- Hide unwanted lock events from internal locks
%   TreeHidden = TreeInternal \ Union(diff({|lockAcquired|}, {|lockAcquired.L.0|}), 
%                                      diff({|unlockedLock|}, {|unlockedLock.L.0|}),
%                                      diff({|callLock|}, 
%                                             {callLock.L.y.T.y.z | T.y.z <- ThreadID}))
%   -- Rename linearization points so all refer to L.0
%   TreeInternalRenamed = TreeHidden [[callLock.L.x.t <- callLock.L.0.t]]
% \end{cspm}

% \framebox{Add analysis of behaviours with different node locks etc}


\subsubsection{Starvation Freedom} \label{starvation-freedom}

Starvation freedom is a liveness property that states that, under the assumption that no thread holds the lock indefinitely, every thread that attempts to acquire the lock eventually succeeds \cite{TAoMP}. It requires that any thread attempting to gain the lock must can only be bypassed by other threads a finite number of times.

One common approach to checking infinite properties in CSP is to hide some (in this case internal) channels and then check that the model does not diverge. This approach does not work here: consider a starvation-free lock which uses busy waiting (repeatedly testing if the lock is available). We have that hiding the internal communications results in a divergence, however the lock is starvation-free. An example of such a lock is the Peterson lock \framebox{link to model}.

Roscoe and Gibson-Robinson showed that every infinite traces property that can be captured by CSP refinement can also be captured by a finite-traces refinement check when combined with the satisfaction of a deterministic B{\"u}chi automaton \cite{RoscoeBuchi}. A B{\"u}chi automaton, as explained by \framebox{REF}, is an automaton that takes infinite inputs and accepts an input if an accepting state is visited infinitely often \framebox{Improve}.\framebox{need to go over the above} However, we can show that no deterministic B{\"u}chi automaton can capture starvation freedom. 

% The automaton should be satisifed by any trace where some thread obtains the lock; this requires that the BA has some accepting state that can be visited an infinite amount of times when some thread |t| holds the lock. Since BA have a finite number of states, that state must be reachable in a finite sequence of steps. As a result, the BA must be satisfied by some thread obtaining the lock, reaching an accepting state in the BA, releasing the lock and repeating that same process. This, however, would accept non-starvation free locks.

% We need at least one accepting state where some thread holds the lock. Otherwise the automaton would reject a starvation free lock with some trace where a thread obtains the lock and never releases it. At least one of these accepting states must be visitable an infinite number of times.

% We can't have an accepting state where some thread holds the lock. Consider a lock that is not starvation free and suppose one thread (wlog.~|T.0|) calls the lock. Some other thread (wlog.~|T.2|) calls the lock, acquires it and releases it without the original caller making any progres

% We have that the BA should be satisfied by a run where some thread immediately obtains a lock and never releases it. We can therefore conclude that any BA must have at least one accepting state where some thread holds the lock and that some subset of the accepting states can be reached infinitely often.

% Now suppose we have a 2-threaded lock which is not starvation free. 

% Suppose we have a 2-threaded lock which is starvation free. The BA should be satisfied by a run where thread |T.1| proceeds and is permanently descheduled before obtaining the lock 

We will show by contradiction that no such deterministic automaton $B$ can capture starvation freedom. Let us without loss of generality consider 2-threaded locks. 

By the definition of starvation freedom, the automaton should be satisfied if |T.0| acquires and never releases the lock and |T.1| attempts to acquire the lock. The automaton therefore must have some accepting state which is visited an infinite number of times when |T.0| holds the lock and |T.1| attempts to acquire it; we will call this state $q_{a}$.

%the repeated actions of the threads cause (a subset of) the accepting states to be visited infinitely often.

% Now suppose that |T.0| again holds the lock and |T.1| is attempting to acquire the lock. By the above, we have that there is some reachable accepting state in the automaton where this is the case; we run until we reach this state and then |T.0| releases the lock. |T.0| then reobtains the lock successfully - we have that the automaton can not block this behviour, else it excludes the starvation-free lock where |T.1| can only obtain the lock when |T.0| has acquired and released the lock an even number of times. Again, since |T.0| could hold the lock indefinitely, we have that some accepting state must be reachable where |T.0| holds the lock and |T.1| is attempting to acquire the lock. Once we reach this state, |T.0| can release the lock and repeat this cycle. Since this cycle can be repeated indefinitely, we have that this is required to be an accepting run if the lock is starvation free. However, we have that the run is accepted even if the lock is not starvation-free; an example being the TAS lock from before as the only communication |T.1| makes whilst trying to obtain the lock unsuccessfully is via same |getAndSet.l.T.1.true.true| communication). 

% We therefore have that we have shown by construction that a general, deterministic B{\"u}chi automaton for starvation-freeness cannot exist.

Now we consider the TAS lock from before. This is clearly not starvation free as the first thread to communicate on |getAndSet| once the lock becomes available will acquire it; this allows some thread to infinitely bypass some other waiting thread. Consider an execution where |T.0| repeatedly acquires and releases the lock, bypassing |T.1| which is repeatedly attempting and failing to obtain the lock. This execution, however, passes through the state $q_{a}$ of $B$ every time |T.0| acquires and later releases the lock. This can occur an infinite number of times, resulting in $B$ accepting this execution. This is a contradiction as this execution is clearly not starvation free as |T.1| never obtains the lock, however it still satisfies $B$.


% Now we consider the TAS lock from before. This is clearly not starvation free as the first thread to communicate on |getAndSet| once the lock becomes available will acquire it. As a result |T.0| can bypass |T.1| an infinite number of times, hence the TAS lock is not starvation free. We have from the above that there must exist some accepting state in the automaton which is reachable when |T.0| holds the lock and |T.1| is attempting to obtain the lock. Since |T.1| only has one available state when |T.0| holds the lock (communicating a |getAndSet.L.0.T.1.false.false|), we have that this accepting state must be reachable whenever |T.0| bypasses |T.1|. As a result, we have that |T.0| can obtain the lock an infinite number of times, and each of these bypasses can result in a visit to an accepting state of the automaton. We therefore have that the B{\"u}chi automaton accepts this run despite the both the run and the TAS lock not being starvation-free. This is a contradiction and hence we have that no such deterministic automaton can exist.

\framebox{Improve}We hence have that no deterministic B{\"u}chi automaton can accurately capture starvation freedom and hence we cannot test directly for starvation-freedom through a standard FDR refinement check.

\subsection{First-come-first-served}

We can instead consider capturing the stronger propety of first-come-first-served. Locks that satisfy this property can be split into a \emph{doorway} section of a bounded number of steps and a folowing \emph{waiting} section of a potentially unbounded number of execution steps. This property states that once thread |X| has completed the doorway section of the lock it cannot be overtaken by a thread |Y| that has not yet started its doorway section; i.e.~ |X| will acquire the lock before |Y| acquires the lock. This implies starvation-freedom as once |X| has completed its doorway section, it can only be bypassed by threads who have started their doorway section prior to |X|.

\begin{cspm}
  FCFS()
\end{cspm}


 %\section{Thoughts on starvation freedom (not intended as part of text)}
Using the definition of starvation freedom from \cite{TAoMP} 
\begin{quote}
"The starvation-freedom property guarantees that every thread that calls lock()
eventually enters the critical section, but it makes no guarantees about how long
this may take."
\end{quote}

To model starvation freedom in the traces model we'd need to be able to force any thread to perform multiple actions; otherwise every deadlock free lock implementation allows one thread to just continually acquire and release the lock and this is clearly starvation free for any thread that doesn't call lock. As a result, we need to use the failures model 
\begin{flushleft}
$\forall(s,ref) \in failures(P), \forall(t) \in ThreadID \, . \,\, $

$\;\; ((\#(s \upharpoonright \{lockAcquired.t\}) + 1 >= \#(s \upharpoonright \{callLock.t\}) >= \#(s \upharpoonright \{lockAcquired.t\})) \wedge$

$\quad (\#(s \upharpoonright \{callLock.t\}) = \#(s \upharpoonright \{lockAcquired.t\}) \vee $ |-- t not waiting for lock|

$\quad\, \:((\#(s \downarrow lockAcquired) = \#(s \downarrow lockReleased)) \implies lockAcquired.t \notin ref) \wedge $ %|--lock currently available|

%$\qquad lockAcquired.t \notin ref \wedge$ |-- t can acquire the lock|

$\quad\, \:\, \: (\forall s' \, st. \,\, (s\,\hat{ }\,s', ref') \in failures(P) \,.\,\, $

||$\qquad \, \:$|--t must acquire lock 1 time more than it calls it in s' in order to terminate correctly|

$\qquad \, \:(\#(s' \upharpoonright \{lockAcquired.t\}) \neq 1 + \#(s' \upharpoonright \{callLock.t\}) \implies ref' \neq \Sigma) \wedge$

$\qquad \, \: (\exists a \leq s' . \,\, (a\, \hat{ }\,\langle lockObtained.t\rangle) \leq s')$

\end{flushleft}

I believe the above works as a specification for starvation freedom as it asserts that |callLock.t| occurs at most once more than |lockObtained.t| in $s$ (correct behaviour of the lock). We also have that either |t| is not trying to acquire the lock at the  end of $s$ or that for any finite trace $s'$ st. $(s\hat{ }s', ref') \in failures(P)$, we have that if $ref' = \Sigma$ i.e.~the system has terminated, then $s'$ must contain exactly one more |lockObtained.t| than |callLock.t| in order to terminate without |t| still waiting for the lock. I believe the last line asserts that for infinite $s'$, there exists some finite prefix $a$ such that $(a\, \hat{ }\,\langle lockObtained.t\rangle) \leq s')$, but am not certain of if there is a way to assert that a is finite (or if indeed this is even necessary). Regulator processes for these individual requirements seem obvious except for last line, that |lockObtained.t| occurs in some finite prefix of $s'$. A process |C(t) = P(0) [> lockObtained.t -> STOP| where |P(n, t) = [] a: LockChans @ a.L.0?s: diff(ThreadID, {t}) -> P(n+1, t)| restricts to countably infinite prefixes, but I can't find a unartificial way of restricting to finite prefixes (are traces restricted in length in FDR?).

If this is not possible then we can resort to capturing the stronger property of first-come-first-served, which in turn implies starvation-freedom. We do this by extending |LockEvents| to |FCFSLockEvents = union(LockEvents)|. We then construct |FCFSActualSystemR = ActualSystemR [[*doorwayEvent?t* <- doorwayComplete.t]]| and |FCFSActualSystemRExtDiv = FCFSActualSystemR \ InternalChannels|.

The following is a pretty scruffy implementation of a Peterson lock (to be tidied up, but more as a proof of concept here):
\begin {cspm}
  datatype Index = I.{0..2}
  -- I.0 and I.1 corresponds to the flags of T.0 and T.1 respectively
  -- I.2 corresponds to the victim variable, with true corresponding 
  --     to T.1 and false corresponding to T.2
  Variables = Array(Index, false, get, set, getAndSet)

  Lock :: (ThreadID) -> Proc
  Lock(T.x) = if x == 1 then set.I.1.T.1.True -> set.I.2.T.1.True -> WhileLock(T.1)
              else if x == 0 then set.I.0.T.0.True -> set.I.2.T.0.False -> WhileLock(T.0)
              else DIV -- only works for two threads
  WhileLock(T.x) = get.I.1-x.T.x?v -> if v == False then SKIP -- lock acquired
                                      else get.I.2.T.x?y ->
                                        if x == 1 and y == True then WhileLock(T.x)
                                        else if x == 0 and y == False then WhileLock(T.x)
                                        else SKIP

  Unlock :: (ThreadID) -> Proc
  Unlock(T.x) = set.I.x.T.x.False -> SKIP

  Thread(T.x) = callLock.L.0.T.x -> Lock(T.x); Unlock(T.x); Thread(T.x)
\end{cspm}

We also define a process |FCFS| which is a trace specification for first-come-first-served. The first parameter is a set containing the nodes that are trying to acquire the lock and have hence started their doorway; the second set contains the nodes that have finished their doorway section but have not yet acquired the lock; the third set contains the live threads. We have that the |FCFSActualSystemRExtDiv| given by the Peterson lock implementation above is trace refined by |FCFS| as defined below. This shows that the Peterson lock is first-come-first-served and hence starvation free. 

We can hence use this refinement check on the tree lock with Peterson locks at each fork in the tree.

I haven't been able to think of/find any lock implementations that are starvation free but not first-come-first-served, but at the very least this gives us a way of proving starvation-freedom for some locks

\begin{cspm}
  -- Works for 2 threads
  FCFS({}, {}, TS)     = callLock.L.0?t -> FCFS({t}, {}, TS)
                        [] end?t:TS -> FCFS({}, {}, diff(TS, {t}))
  FCFS({t}, {}, TS)    = callLock.L.0?t2:diff(TS, {t}) -> 
                              FCFS(union({t}, {t2}), {}, TS)
                        [] doorwayComplete.L.0.t -> FCFS({}, {t}, TS)
                        [] end?t2:diff(TS, {t}) -> FCFS({t}, {}, diff(TS,{t2}))
  FCFS(ts, {}, TS)     = doorwayComplete.L.0?t:ts -> 
                          (doorwayComplete.L.0?t2:diff(ts, {t}) -> 
                              lockAcquired.L.0?t3:ts -> 
                                FCFS({}, diff(ts, {t3}), TS)
                            [] lockAcquired.L.0.t -> FCFS(diff(ts, {t}), {}, TS))
  FCFS({}, {t}, TS)    = lockAcquired.L.0.t -> FCFS({}, {}, TS)
                        [] callLock.L.0?t2:diff(TS, {t}) -> 
                              (lockAcquired.L.0.t -> FCFS({t2}, {}, TS) 
                              [] doorwayComplete.L.0.t2 -> 
                                    lockAcquired.L.0.t -> FCFS({}, {t2}, TS))
                        [] end?t2:diff(TS, {t}) -> FCFS({}, {t}, diff(TS, {t2}))
  FCFS({t1}, {t2}, TS) = lockAcquired.L.0.t2 -> FCFS({t1}, {}, TS)
                        [] doorwayComplete.L.0.t1 -> lockAcquired.L.0.t2 
                                                  -> FCFS({}, {t1}, TS) 

  assert FCFS({}, {}, ThreadID) [T= FCFSActualSystemRExtDiv \ {|lockReleased|}
\end{cspm}

\subsubsection{$\omega$-regular languages}

\framebox{Currently incorrect}

Starvation freedom can be quite easily expressed as a $\omega$-regular language. Indeed, if we rename all internal actions in |ActualSystemR| to |wait| and hide all |callLock| and |lockAcquired| events from threads other than |t|, the lock is starvation free for thread |t| if it can produce all traces of the following $\omega$-regular language:
\begin{cspm}
  ((£$\neg$£(callLock.t))£$^{*}$£.(callLock.t).(wait)£$^{*}$£.(lockAcquired.t))£$^{\omega}$£ + 
  ((£$\neg$£(callLock.t))£$^{*}$£.(callLock.t).(wait)£$^{*}$£.(lockAcquired.t))*.(£$\neg$£(callLock.t))£$^{\omega}$£
\end{cspm}

We can therefore also model starvation freedom using B{\"u}chi automata, which accepts infinite traces that reach an accepting state an infinite number of times




 %\inlineScala
\section{Monitors}
\framebox{This is intended to be used in an earlier background section}

A monitor can be used to ensure that certain operations on an object can only be performed under mutual exclusion. Here we first consider the implementation of a monitor used by the Java Virtual Machine (JVM), before considering an alternative implementation that addresses some of the limitations of the JVM monitor.

\subsection{The JVM monitor}

Mutual exclusion between function calls is provided inside the JVM via |synchronized| blocks. Only one thread is allowed to be active inside the synchronized blocks of an object at any point; a separate thread trying to execute a |synchronized| expression will have to wait for the former to release the lock before proceeding. Inside a |synchronized| block, a thread can also call |wait()| to suspend and give up the lock. This waits until a separate thread (which can now proceed) executes a |notify()|, which will wake the waiting thread and allow it to proceed once the notifying thread has released the lock.

It is important to note that the implementation of |wait()| is buggy. Sometimes a thread that has called |wait()| will wake up even without a |notify()|; this is called a \emph{spurious wakeup}.




\subsubsection{Modelling the JVM monitor}

\inlineCSP

For our model of a monitor in CSP we have extended the |JVMMonitor| provided by Lowe \cite{LoweJVMMonitor}.

Lowe's module previously provided a single monitor; this is problematic in case we have multiple objects which each require their own monitor. We instead introduce a datatype |MonitorID|, with this type containing all possible 'objects' that could require their own monitors. The |JVMMonitor| module is then changed to be parameterised over some subset of |MonitorID|. The internal channels and processes now also take some |MonitorID| value to identify which object is being reffered to at any point.

%This required changing the module to parameterise it over the |Monitor| type, with this containing all possible 'objects' that could require a monitor, allowing for each object we model to have its own, independent monitor; this is similar to how the JVM assigns each object its own monitor. The original implementation requires that a new |JVMMonitor| be instantiated every time a unique monitor was required; this is very fiddly for when we have a variable number of monitors required. Instead the internal channels are now of the type \inlineCSP |Monitor . ThreadID|, allowing for easy identification of which object the monitor accompanies.

Internally, the model uses one lock per |MonitorID|. These also have an additional parameter storing the identities of any waiting threads\framebox{full code in an appendix?}. So that it is a faithful model of an actual JVMMonitor we also model spurious wakeups via the |spuriousWakeup| channel. We therefore run the the lock process in parallel with a regulator process |Reg = CHAOS({|spuriousWakeup|})|; this is used to non-deterministically allow or block spurious wakeups where appropriate. This is important as when we hide |spuriousWakeup| events we have that every state in FDR that allows a |spuriousWakeup| has a pair which blocks the |spuriousWakeup|. This allows us to perform refinement checks in the stable-failures model instead of just the failures-divergences model, allowing us to write more natural specification processes.

Our model of the monitor provides the following exports:

\begin{cspm}[caption={The interface of the JVMMonitor module; changes are underlined}, label={listing::JVMMonitorInterface}]
module JVMMonitor"(MonitorID)"
    ...
    Reg = CHAOS({|spuriousWakeup|})
exports

    -- All events except spuriousWakeup
    events = {| acquire, release, wait, notify, notifyAll |}

    channel spuriousWakeup : "MonitorID" . ThreadID
  
    "InitialiseAll ="
      "||| mon <- MonitorID @ (Unlocked(mon, {})  [| {|spuriousWakeup|} |] Reg)"

    runWith("obj", P) = P [| events |] (Unlocked("obj", {})  [| {|spuriousWakeup|} |] Reg)

    "runWithMultiple(objs, P) ="
    "P [| events |] (||| obj <- objs @ (Unlocked(obj, {})  [| {|spuriousWakeup|} |] Reg))"

    -- Interface to threads.

    -- Lock the monitor
    Lock("obj", t) = ...

    -- Unlock the monitor
    Unlock("obj", t) = ...

    -- Perform P under mutual exclusion
    Synchronized("obj", t, P) = ...

    -- Perform P under mutual exclusion, and apply cont to the result. 
    -- MutexC :: (ThreadID, ((a) -> Proc) -> Proc, (a) -> Proc) -> Proc
    SynchronizedC("obj", t, P, cont) = ...

    -- Perform a wait(), and then regain the lock.
    Wait("obj", t) = ... 

    -- perform a notify()
    Notify("obj", t) = ...

    -- perform a notifyAll()
    NotifyAll("obj", t) = ...

endmodule
\end{cspm}



Each monitor provides |Wait(obj, t)|, |Notify(obj, t)| and |Synchronized(o, t, Proc)| methods to model the equivalent functions/blocks in SCL. The |Proc| parameter is used to specify the process that will be run inside the |Synchronized| block; the intended usage of this is of the form |callFunc.o.t -> Synchronized(o, t, syncFunc)| where the process communicates that it is calling the model of fucntion |Func| before completing the rest of the function whilst holding the monitor lock. |runWith(obj, threads)| and |runWithMultiple(objs, threads)| are used to initalise a single monitor with identity |obj| or multiple monitors with identities |objs| respectively to synchronise threads that interact with a single object and multiple objects respectively.

% \subsection{|LockSupport|}

% \lstinputlisting{C:\Users\tom\Documents\repos\4yp\new-csp\lock-support-module.csp}


 %\subsection{The SCL Monitor}


There are two main limitations to the standard JVM monitor: it suffers from spurious wakeups and does not allow targeting of |notify| calls. Spurious wakeups are obviously bad and are a common source of bugs where not adequately protected against. Targeting of signals can also be very beneficial to the performance of a program; take for example the one-place buffer shown below in Listing \ref{code::slot}. Suppose we have significantly more threads wanting to |get| a value than |put| a value. Each |notifyAll| will awake every thread waiting on a |get|, even if |get|s are blocked by |! available|. This results in the majority of the threads immediately sleeping again, adding significant overhead. A |notifyAll| is also required since the JVM monitor makes no guarantees as to which thread is awoke by a |notify|; as a result repeated |notify|s could potentially just wake up two thread alternatively, both of which are waiting to perform the same process.

% \begin{lstlisting}[language=Scala, label=code::slot, caption={Single placed buffer as an example of the inefficiency of untargeted signals}]
\begin{scala}[label=code::slot, caption={Single placed buffer as an example of the inefficiency of untargeted signals}]
  class OneBuff[T] {
    private var buff = null.asInstanceOf[T]
    private var available = false

    def put(x: T) = synchronized {
      while(available) wait()
      buff = x
      available = true
      notifyAll()
    }

    def get : T = synchronized {
      while(! available) wait()
      available = false
      notifyAll()
      return x // not overwritable as we still hold the lock
    }
  }
\end{scala}

The SCL monitor implementation solves both of these issues with a single monitor offering multiple distinct |Conditions| to allow for more targeted signalling. It is worth noting that these improvements result in the SCL monitor being more computationally expensive per call\framebox{Improve sentence}. Each individual condition offers the following operations:
\begin{itemize}
  \item |await()| is used to wait for a signal on the condition
  \item |signal()| is used to signal to a thread waiting on the condition
  \item |signalAll()| is used to signal to all the threasd waiting on the condition
\end{itemize}
Each of these operations should be performed while holding the lock. We can note that these operations are similar to the JVM monitor |wait()|, |notify()| and |notifyAll()| respectively. This functionality is also similar to the |java.util.concurrent.locks.Condition| class; the primary difference is that the SCL monitor blocks spurious wakeups whereas they are allowing by the JAVA class.

Considering the single-placed buffer, we can use two conditions to separate the threads attempting to |get| and those trying to |put|. We can then modify the program above to only perform a single |signal| towards the threads that are attempting to perform the opposing function, resulting in significant efficiency gains as no threads need to immediately sleep after being woken up.

The implementation of the SCL Monitor, available on GitHub\footnote{\url{https://github.com/GavinLowe1967/Scala-Concurrency-Library/blob/main/src/Lock/Lock.scala}}, makes use of the Java |LockSupport| class; we explore this in the next section\framebox{Sub?}. Here we present a low-level model of an SCL Condition which makes use of a model of the |LockSupport| class.

\subsection{LockSupport}

Threads interact with the |LockSupport| module via three main events: a thread can park itself, a thread can unpark another thread and a parked thread can wake up. We therefore introduce channels |park|, |unpark| and |wakeUp| to represent these three synchronisations. 

It is also important to note that |LockSupport| is also affected by spurious wakeups. We therefore add a Boolean parameter to the |wakeUp| channel, using |false| to indicate a spurious wakeup and |true| otherwise.

\begin{cspm}[caption={The CSP model of the Java LockSupport module}]
  module LockSupport

    channel park: ThreadID
    channel unpark: ThreadID.ThreadID
    channel wakeUp: ThreadID.Bool

    LockSupport :: ({ThreadID}, {ThreadID}) -> Proc
    LockSupport(waiting, permits) =
      if waiting == {} then LockSupport1(waiting, permits)
      else (    LockSupport1(waiting, permits)
            |~| wakeUp$t:waiting!false -> LockSupport(diff(waiting, {t}), permits))

    LockSupport1(waiting, permits) =
      park?t-> (
        if member(t, permits)
          then wakeUp.t.true -> LockSupport(waiting, diff(permits, {t}))
        else LockSupport(union(waiting, {t}), permits) )
      []
      unpark?t?t2-> ( 
        if member(t2, waiting)
          then wakeUp.t2.true -> LockSupport(diff(waiting, {t2}), permits)
        else LockSupport(waiting, union(permits, {t2})))
    

    LockSupportDet :: ({ThreadID}, {ThreadID}) -> Proc
    LockSupportDet(waiting, permits) = LockSupportDet1(waiting, permits)
    LockSupportDet1(waiting, permits) = ... -- Analogous to LockSupport1

  exports

    InitLockSupport = LockSupport({}, {})

    InitLockSupportDet = LockSupportDet({}, {})

    Park(t) = park.t -> wakeUp.t?_ ->  SKIP

    Unpark(t, t') = unpark.t.t' -> SKIP


  endmodule _*$*_
\end{cspm}

Internally, the module stores two sets of threads: those parked and those with permits available. A thread that is parking is either added to the waiting set if no permit is available or it is immediately re-awoken. When there is at least one thread waiting and no ongoing |wakeup|, the |LockSupport| module can nondeterministically choose to either operate as normal or to allow one of the waiting threads to spuriously wakeup.

|InitLockSupportDet| is defined similarly to |InitLockSupport|; the only change is made by removing the nondeterministic choice for a thread to spuriously wakeup. This will allows us to show some divergence results later.

\subsection{The SCL monitor model}

\framebox{DIAGRAM}

We now consider our model of the SCL monitor. This consists of three main components:
\begin{itemize}
  \item The monitor lock; this is a simple process |Lock(m) = acquire.m?t -> release.m.t -> Lock(m)| which specifies that only one thread can hold the lock and that same thread must release the lock before some other thread can obtain it.
  \item The |LockSupport| module; this is as described above.
  \item The queue of |ThreadInfo| values.
\end{itemize}

\subsubsection{The ThreadInfo Queue}

In the Scala code, each |Condition| maintains a queue of |ThreadInfo| values which have a thread id and a variable |ready| indicating whether the corresponding thread has been unparked. The natural method of modelling this queue in CSP is with a series of nodes, a series of processes each corresponding to a node and some co-ordinator processes.

Since we are handling |n| threads and all of them can be waiting at the same time, we hence need |n| separate nodes, which we will represent as |datatype Node = N.{0..n-1}|. Each of the nodes can be initialised by a thread, at which point it then acts as that thread's |ThreadInfo| object. It allows threads (including its allocated thread) to check the value of |ready| via a communication on the |isReady| channel, and for other threads to set the value of |ready| to true via the |setReady| channel. Once the parent thread has reawoken legitimately (i.e.~via an |Unpark| not a spurious wakeup) it then releases the node; this is valid as the node must have already been dequeued and |setReady|, hence no further communications should happen until it is reinitialised. The |ThreadInfo| objects are designed to diverge whenever a communcation occurs that should not be possible in the original Scala code.

\begin{cspm}
  ThreadInfo :: (Node) -> Proc
  ThreadInfo(n) = 
       initialiseNode.n?t -> ThreadInfoF(n, t)
    [] isReady.n?t?t2.true -> DIV
    [] setReady.n?t?t2 -> DIV
  ThreadInfoF(n, t) =  
       isReady.n!t?t2.false -> ThreadInfoF(n, t)
    [] setReady.n!t?t2:diff(ThreadID, t) -> ThreadInfoT(n, t)
    [] initialiseNode.n!t -> DIV
  ThreadInfoT(n, t) = 
       isReady.n!t?t2.true -> ThreadInfoT(n, t)
    [] setReady.n!t?t2 -> DIV
    [] initialiseNode.n!t -> DIV
    [] releaseNode.n.t -> ThreadInfo(n)
\end{cspm}


We then use a process called |NodeAllocator| to allocate the |Nodes| to threads and also to collect them when they are no longer required.
We note that any thread can use any node in this model; this is analogous to the nodes being memory chunks allocated to each thread by |NodeAllocator| and then garbage collected once they are no longer needed.

\begin{cspm}
  NodeAllocator(ns) = 
    (not(empty(ns))) & 
      (initialiseNode$n:ns?t -> NodeAllocator(diff(ns, {n})))
  [] releaseNode?n:ns?t -> DIV
  [] releaseNode?n:diff(Node, ns)?t -> NodeAllocator(union(ns, {n}))_*$*_
\end{cspm}

Finally we have the |Queue| processes, which model the queues maintained inside each Condition. Each |Queue| keeps a sequence of the nodes waiting on its condition, with each Node corresponding to its current holding thread.

\begin{cspm}
  Queue'(m, c, qs) = 
       (not(null(qs))) & dequeue.m.c?t!head(qs) -> Queue'(m, c, tail(qs))
    [] (null(qs)) & isEmpty.m.c?t -> Queue'(m, c, qs)
    [] enqueue.m.c?t?n:diff(Node, QS) -> Queue'(m, c, qs^<n>) 
\end{cspm}

We have that each Queue' is always ready to accept an |enqueue| (unless all nodes are already in the queue), but will only communicate one of |dequeue| or |isEmpty| at any point in time. We note that the restriction on the values of |n| that can be enqueued is such that the queue is of finite length; this is required for efficient model checking in FDR.

\subsection{The functions and interface of the monitor}

Now we have the components of the model of the monitor, the last step is to place these processes in parallel. The majority of these processes are independent of each other; only the |NodeAllocator| and the |ThreadInfo| processes need to synchronise with each other, which occurs when a node is either initialised or released.
\begin{cspm}
  InitialiseMon(m, setC) = 
    (Lock'(m) ||| 
     (||| c <- setC @ Queue(m, c, <>)) |||
     (NodeAllocator(Node) [|{|initialiseNode, releaseNode|}|] 
        (||| n <- Node @ ThreadInfo(n))) ||| 
     InitLockSupport)
\end{cspm}

The version of the monitor without spurious wakeups, |InitialiseMonDet(m, setC)|, is defined similarly but interleaved with |InitLockSupportDet(m, setC)| instead. 

The first two processes we export as part of the interface of our monitor are |runWith(P, mon, setC)| and |runWithDet(P, mon, setC)|, which each take a number of threads in |P|, an identity for the monitor and a set of conditions on that monitor. These processes synchronise the processes in |P| with the |InitialiseMon(m, setC)| and |InitialiseMonDet(m, setC)| respectively. This is to allow the threads to `call' the various functions that act on the monitor correctly and so that mutual exclusion can be enforced as intended.

\begin{cspm}
  -- The set of events that are hidden when a thread uses the monitor
  HideSet(m, setC) = 
    {|park, unpark, wakeUp, enqueue.m.c, dequeue.m.c, initialiseNode, nodeThread,
      setReady, isReady, isEmpty.m.c, releaseNode | c <- setC|}

  -- The set of events to synchronise on between a thread and the monitor
    SyncSet2(m, setC) = Union({{|acquire.m|}, {|release.m|}, HideSet(m, setC)})
  
  exports

  channel acquire, release: MonitorID.ThreadID
  channel callAcquire, callRelease: MonitorID.ThreadID
  channel callAwait, callSignalAll: MonitorID.ConditionID.ThreadID
  channel callSignal: MonitorID.ConditionID.ThreadID

  -- Runs the monitor with internal spurious wakeups
  runWith(P, mon, setC) = 
    ((P [|SyncSet(mon, setC)|] 
          InitialiseMon(mon, setC)) \ HideSet(mon, setC))

  -- Runs the monitor without internal spurious wakeups
  runWithDet(P, mon, setC) = 
    ((P [|SyncSet(mon, setC)|] 
          InitialiseMonDet(mon, setC)) \  HideSet(mon, setC))

  ...
\end{cspm}

In the definitions above, |SyncSet| contains every event that we need to synchronise on between a series of threads and the monitor. We then hide all events except for those representing a thread acquiring and releasing the lock; this is the contents of |HideSet|.

We now consider the functions offered by the monitor. We have the interface given below, with each of the five processes corresponding to the function of the same name. Each process starts with a communication on the correspondingly named |callX| channel to indicate that the specified thread has just called that function; this makes examining any traces produced significantly simpler. We will refer to these |callX| communications as `external' and all other channels as being `internal'. \framebox{Better convention for placeholder values?}
  
\begin{cspm}
export 
  ...

  -- Operations on the monitor
  Await(mon, cnd, t) = callAwait.mon.cnd.t -> ...

  Signal(mon, cnd, t) = callSignal.mon.cnd.t -> ...

  SignalAll(mon, cnd, t) = callSignalAll.mon.cnd.t -> ...

  Lock(mon, t) = callAcquire.mon.t -> acquire.mon.t -> SKIP

  Unlock(mon, t) = callRelease.mon.t -> release.mon.t -> SKIP

 endmodule
\end{cspm}


Both |Lock| and |Unlock| both only require a single internal communication (either acquiring or releasing the lock) after the thread's external communication. By contrast |Await|, |Signal| and |SignalAll| are more complex, so the initial external communication is followed by another process, in each case named |X1|. Each of these processes are natural translations of the Scala code into CSP; the main exception is using |Await2| to represent the |while| loop in the Scala |await()| function.

\begin{cspm}
  Signal1(mon, cnd, t) = 
       isEmpty.mon.cnd.t -> SKIP
    [] dequeue.mon.cnd.t?n -> nodeThread.n?t2!t -> isReady.n.t2.t?b ->
         (if b then Signal1(mon, cnd, t)
          else setReady.n.t2.t -> Unpark(t, t2); SKIP)

  SignalAll1(mon, cnd, t) =
       isEmpty.mon.cnd.t -> SKIP
    [] dequeue.mon.cnd.t?n -> nodeThread.n?t2!t -> setReady.n.t2.t -> 
            Unpark(t, t2); SignalAll1(mon, cnd, t)

  Await1(mon, cnd, t) = 
    initialiseNode?n!t -> enqueue.mon.cnd.t.n -> release.mon.t -> Await2(mon, cnd, t, n)

  Await2(mon, cnd, t, n) = 
    isReady.n.t.t?b -> (if b then releaseNode.n.t -> acquire.mon.t -> SKIP
                        else Park(t); Await2(mon, cnd, t, n))
\end{cspm}



\subsection{Correctness}

We now consider the correctness of our model. We present a specification process for a idealised monitor with conditions and perform refinement checks against it. We show that the ordering of awaits is also upheld by a separate  refinement check.
  
  We will first consider the specification process of a monitor with multiple conditions. Each of the monitor processes are parametised over the identity of the monitor, a map of |ConditionID => {ThreadID}| representing the set of threads waiting on each |Condition|, and a set of |ThreadID|s that are waiting to obtain the lock. We choose to use sets of waiting threads instead of queues of waiting threads to make this a more general specification of a monitor; we consider orderings in a later test. |initSet| is the initial mapping of the waiting threads, with each condition mapping to an empty set. We define |valuesSet| as a helper function which returns a set of all the threads that are currently waiting on any condition; this allows us to restrict the specification to only allow threads that aren't waiting to obtain the lock. 
  
  We also define a new channel |callSignalSpec|. This is similar to the |callSignal| channel introduced earlier, but has an additional parameter indicating which thread is being signalled. A thread will `signal' itself if no threads are waiting on the selected condition, otherwise it will non-deteministically signal one of the waiting threads. This extra parameter is required as there is no set operator to select a single element of a set in CSP; we use this channel instead to indicate the selected |ThreadID| for signalling. We rename |callSignalSpec| back to |SCL::callSignal| for when we perform the refinements.

  \begin{cspm}
initSet = mapFromList(<(c, {}) | c <- seq(ConditionID)>)
values(map) = Union({mapLookup(map, cnd) | cnd <- ConditionID})
channel callSignalSpec: MonitorID.ConditionID.ThreadID.ThreadID

SpecUnlocked(m, waiting, poss) =
     SCL::callAcquire.m?t':diff(ThreadID, union(values(waiting), poss)) ->
       SpecUnlocked(m, waiting, union(poss,{t'}))
  [] SCL::acquire.m?t:poss -> SpecLocked(m, t, waiting, diff(poss,{t}))
      

SpecLocked(m, t, waiting, poss) =
  [] c': ConditionID @  
      (   
          (mapLookup(waiting, c') == {}) & callSignalSpec.m.c'.t.t ->
             SpecLocked(m, t, waiting, poss)
       [] (mapLookup(waiting, c') != {}) & 
            callSignalSpec.m.c'.t?t':mapLookup(waiting, c') -> 
              SpecLocked(m, t, 
                        mapUpdate(waiting, c', diff(mapLookup(waiting, c'), {t'})), 
                        union(poss, {t'}))
      )
  [] SCL::callSignalAll.m?c:ConditionID!t -> 
       (if mapLookup(waiting, c) == {} then 
            SpecLocked(m, t, waiting, poss) 
        else SpecLocked(m, t, mapUpdate(waiting, c, {}), 
                          union(poss, mapLookup(waiting, c))))
  [] SCL::callRelease.m.t -> 
        SpecLockedReleasing(m, t, waiting, poss)
  [] SCL::callAcquire.m?t':diff(ThreadID, 
                                Union({values(waiting), poss, {t}})) -> 
        SpecLocked(m, t, waiting, union(poss, {t'}))
  [] SCL::callAwait.m?c:ConditionID!t -> 
        SpecLockedWaiting(m, c, t, waiting, poss)
  \end{cspm}

  Here we have defined the processes where either the monitor lock is not held, or where it is held by thread |t| and is waiting to perform a function. We next define the processes where |t| is in the middle of waiting or releasing the lock.

  \begin{cspm}
-- t doing a wait; needs to release the lock
SpecLockedWaiting(m, c, t, waiting, poss) =
      SCL::release.m.t -> 
        SpecUnlocked(m, mapUpdate(waiting, c, union(mapLookup(waiting, c), {t})), 
                    poss)
  [] SCL::callAcquire.m?t':diff(ThreadID, 
                                Union({values(waiting), poss, {t}})) ->
        SpecLockedWaiting(m, c, t, waiting, union(poss, {t'}))

-- t releasing the lock
SpecLockedReleasing(m, t, waiting, poss) =
      SCL::release.m.t -> SpecUnlocked(m, waiting, poss)
  [] SCL::callAcquire.m?t':diff(ThreadID, 
                                Union({values(waiting), poss, {t}})) ->
        SpecLockedReleasing(m, t, waiting, union(poss, {t'}))

SpecSCL = (let m = SigM.S.0 within 
            (SpecUnlocked(m, initSet, {}) 
              [[callSignalSpec.m.c.t.t' <- SCL::callSignal.m.c.t 
                  | c <- ConditionID, t <- ThreadID, t' <- ThreadID]]))
  \end{cspm}

We first note that the specification process provided is divergence free; we choose this as an idealised monitor should never internally diverge. 

To test against this specification, we interleave a number of process of |ThreadSCL(t)|, with each of these representing the potential (correct) usage of the monitor that thread |t| could perform. These are interleaved to form |ThreadsSCL| and then this is then synchronised with the SCL monitor, via the use of |runWith| or |runWithDet| as outlined above.

\begin{cspm}
ThreadSCL(t) = SCL::Lock(SigM.S.0, t); ThreadSCL1(t)
ThreadSCL1(t) =   
  [] c : ConditionID @ 
     (
          (SCL::Await(SigM.S.0, c, t); ThreadSCL1(t))      
      [] (SCL::Signal(SigM.S.0, c, t); ThreadSCL1(t))
      [] (SCL::Signal(SigM.S.0, c, t); ThreadSCL1(t))
      [] (SCL::SignalAll(SigM.S.0, c, t); ThreadSCL1(t))
     )
  [] (SCL::Unlock(SigM.S.0, t); ThreadSCL(t))

ThreadsSCL = ||| t<-ThreadID @ ThreadSCL(t)

SCLSystem = SCL::runWith(ThreadsSCL, SigM.S.0, ConditionID)
SCLSystemDet = SCL::runWithDet(ThreadsSCL, SigM.S.0, ConditionID)

assert not SCLSystem :[divergence free]
assert SCLSystemDet :[divergence free]
\end{cspm}

We have that both the assertions pass: |SCLSystem| is not divergence free, but |SCLSystemDet| is. Since the only difference between |SCLSystem| and |SCLSystemDet| is that we block spurious wakeups in the latter, we can therefore conclude that divergence is only possible as a result of repeated spurious wakeups of waiting threads. Similarly to \framebox{ref}, we have that this potential divergence is not a major concern since it relies on infrequent spurious wakeups occuring. We also note that, similarly to before in \framebox{ref to previous chapter}, we have that each of these states where a divergence can occur has a corresponding stable state, hence it is valid for us to check refinement under stable-failures in this case.

\begin{cspm}
  assert SpecSCL [F= (SCLSystem) 
  assert SpecSCL [FD= (SCLSystemDet)
\end{cspm}

We have that both the assertions hold, indicating that the SCL monitor fulfils the specification of a monitor as required.

We next consider the fairness of the monitor with regards to individual |signal| calls. In the SCL monitor, queues are used so that each |signal| wakes the thread that has been waiting for the longest time on the condition (if one exists). We test that this property holds using |AwaitOrder|, a process which maintains a list of the threads waiting on each condition in the order that they started waiting.

\begin{cspm}
valuesSeq(map) = Union({set(mapLookup(map, cnd)) | cnd <- ConditionID})
channel error: MonitorID
OrderCheck(m, waiting) = 
      SCL::acquire.m?t:ThreadID -> 
      (if member(t, valuesSeq(waiting)) then error.m -> STOP--DIV
        else OrderCheck(m, waiting))
  [] SCL::callAwait.m?c?t -> 
      (if member(t, valuesSeq(waiting)) then error.m -> STOP
        else OrderCheck(m, mapUpdate(waiting, c, mapLookup(waiting, c)^<t>)))
  [] SCL::callSignalAll.m?c?_ -> 
        OrderCheck(m, mapUpdate(waiting, c, <>))
  [] SCL::callSignal.m?c?_ -> 
      (if null(mapLookup(waiting, c)) then OrderCheck(m, waiting)
        else OrderCheck(m, mapUpdate(waiting, c, 
                                    tail(mapLookup(waiting, c)))))
\end{cspm}

We introduce a new channel |error| here. Any communication on this channel indicates that the ordering of the threads has not been maintained correctly, hence we can use the specification process and refinement checks to establish this. This new process only synchronises on the events that indicate a thread waking, waiting or acquiring the lock; this is sufficient to detect any threads which have non-spuriously woken up before they should.

To run the refinement checks, we place |OrderCheck| in parallel with |SCLSystem| and synchronise on all events that |OrderCheck| offers except for |error.m|. We then check that this still refines |SpecSCL| under stable failures, which it does. We can therefore conclude that no |error| events occur and no new stable failures are introduced, hence the ordering within the model of the SCL monitor are maintained correctly.

\begin{cspm}
assert SpecSCL [F= (OrderCheck(SigM.S.0, initSeq) 
                     [|{|SCL::callAwait.SigM.S.0,
                         SCL::acquire.SigM.S.0,
                         SCL::callSignal.SigM.S.0,
                         SCL::callSignalAll.SigM.S.0|}|] SCLSystem)
\end{cspm}

\subsection{Limitations of natural model of the queue}

Though the model given above is a natural model of the SCL monitor, this is quite ill suited to refinement checking in FDR. The current implementation of the queue allows any thread to obtain and use any of the |Node|s as its own; this leads to exponential blow up in the number of states as the number of threads increases. Considering a case where we have n threads and m are currently waiting with their nodes queued, this has $n\choose{m}$, or $O(n^{m})$ permutations.

We can instead use the same nodes, but restrict them so that each node |N.x| can only be used by the respective thread |T.x|, removing this source of blow up. This is most trivially done by changing |Await1| to specify the node to initialise and not a random one allocated by |NodeAllocator| i.e.~as follows:

\begin{cspm}
  Await1(mon, cnd, t) = initialiseNode.N.t -> ...

  NodeAllocator(ns) = 
      (not(empty(ns))) & (initialiseNode?n:ns?t -> NodeAllocator(diff(ns, {n}))) 
   [] ...
\end{cspm}

We will refer to this version of the queue as the `Simple' model.

For further performance improvements, we can also remove the node allocator process as each node is pre-allocated. Additionally we can change the type signature of |Node| to |N.ThreadID| and simplify many of the channels (removing |nodeThread| and |releaseNode| entirely) as node indicates which thread it corresponds to as follows:

\begin{cspm}
  datatype Node = N.ThreadID
  channel enqueue: MonitorID.ConditionID.Node 
  channel dequeue: MonitorID.ConditionID.ThreadID.Node
  channel setReady: Node.ThreadID
  channel isReady: Node.ThreadID.Bool
  channel initialiseNode: Node
  channel isEmpty: MonitorID.ConditionID.ThreadID
  channel await, signalAll: MonitorID.ConditionID.ThreadID
\end{cspm}

All the definitions remain the same apart from removing any |nodeThread| and |releaseNode| communications and the required type changes\framebox{put raw code in an appendix?}. We keep |initialiseNode| to so that a thread can use it to indicate it is initialising a `new' |ThreadInfo| object and hence to reset the |ready| value to false. We also change |InitialiseMon| and |InitialiseMonDet| to remove the |NodeAllocator|; each of the individual |ThreadInfo| processes are still interleaved as before. We will refer to this as the `optimised' version.

We first need to check that this simplified model remains correct. To complete this, we repeat the same refinement checks as before. These still all pass, indicating that the monitor model with a modified queue fulfills the specification similarly\framebox{wording} to the natural queue model.

We next verify that the efficiency improvements occurs in practice too. We do this by running the FDR verification of |assert SpecSCL [F= SCLSystem| for a range of numbers of threads and conditions. We then compare the number of states generated by the natural queue model against the more efficient queues, with the results visible in table \ref{table::queue}.

\def\thickhline{\noalign{\hrule height 1.5pt}}

\begin{table}
  \renewcommand*{\arraystretch}{1.2}
  \caption{The number of states generated by FDR for the different queue implementations. The improvement value is given as the $\frac{\text{Original number of states}}{\text{Reduced number of states}}$}
    \begin{tabularx}{\linewidth}{|l|l|X|X|X|X|X|}
      \thickhline
      No.&No.& \multicolumn{5}{l|}{Number of states} \\
      threads&conditions& Natural & Simple & Improvement & optimised & Improvement\\
      \thickhline
      2 & 1 & 2288 & 1088 & 2.10& 904 & 2.53\\ \hline
      3 & 1 & 239428 & 36262 & 6.60& 26494 & 9.04 \\ \hline
      4 & 1 & 3.14$\times\text{10}^\text{7}$ & 1180416 & 26.7& 792240 & 39.7\\ \hline
      5 & 1 & 5.39$\times\text{10}^\text{9}$ & 4.06$\times\text{10}^\text{7}$ & 133& 2.59$\times\text{10}^\text{7}$ &208 \\
      \thickhline
      2 & 2 & 4932 & 2382 & 2.07& 2382 & 2.40\\ \hline
      3 & 2 & 686896 & 106672 & 6.44& 82973 & 8.27\\ \hline
      4 & 2 & 1.22$\times\text{10}^\text{8}$ & 4655652 & 26.2& 3363492 & 36.3\\ 
      \thickhline
      2 & 3 & 8436 & 4106 & 2.05& 3634 & 2.32\\ \hline
      3 & 3 & 1445008 & 227512 & 6.35 & 184276 & 7.84\\ \hline
      4 & 3 & 3.15$\times\text{10}^\text{8}$ & 1.22$\times\text{10}^\text{7}$ & 25.9& 9212868 & 34.2\\ 
      \thickhline
    \end{tabularx}
    %\vspace*{5mm}
    \label{table::queue}
  \end{table}

  Here we see that the restricted model with each thread allocated a single node to use results in a state space reduced by a factor of at least $n!$ where $n$ is the number of threads. 
  
  % If we consider the 
  
  
  % This is as expected: we have $n!$ possible allocations of threads to nodes (one per node, and with this weakly associating any unallocated node with some thread without an allocated node). Whenever a node |n| is allocated to thread |t| by an |initialiseNode|, we swap the thread that |n| was associated with to the node that |t| was previously weakly associated to. By contrast, the simplified queue has that each node can only ever be allocated/associated with a single node; there is only 1 permutation for this. As a result, for every single state that the model with the simplified queue can be in, there are $n!$ states of the natural model that are identical in all manners other than the node allocations.

  \framebox{Check exact explanation, also check wrt normalisation and symmetry}
  This is as expected: the simplified queue has one possible bijective mapping of threads to nodes. By contrast, the natural queue has $n!$ bijective mappings of threads to nodes. As a result, for every single state that the model with the simplified queue can be in, there are upto $n!$ states of the natural model that are identical in all manners other than the node allocations.

  Though the state space clearly still grows exponentially with the simplified queues, it is significantly more efficient and makes refinement checks for larger numbers of threads and conditions significantly more feasible.

  \framebox{Introduce efficient spec version of SCL monitor and compare performance?}.




  %If we consider the nodes in the natural implementation, we have that if $n'$ of the $n$ nodes have been allocated, then there are $(n-n')!$ possible allocations of the remaining nodes to the remaining threads. As a result, we can consider each of the $n$ nodes to be `paired' to one of the threads ny point in time, with this allocation 

   


  
 %\newpage
\section{Barrier synchronisation}
\inlineScala

A \emph{barrier synchronisation} object is used to synchronise some number of threads. This allows for a program with threads working on some shared memory which all threads can update to use a number of rounds of synchronisation in order to ensure thread-safety. Programs which use \emph{global synchronisations} (synchronising all threads) typically operate by instantiating some barrier object and then having each thread call |sync| on the barrier once they have completed their current round. Each call to |sync| only returns after all threads have called |sync|, synchronising all the threads at that point in time and allowing the threads to proceed afterwards \cite{CP}.

Here we model and analyse an |n| thread barrier synchronisation object which internally uses a binary heap of |n| two-thread signalling objects.

\subsection{The signalling object}

We first consider the signalling object |Signal|. This is used to synchronise between a `parent' and `child' thread, providing three external methods:

\begin{itemize}
  \item |signalUpAndWait| is used by the child to signal to the parent that the child is ready to synchronise and waits until the parent signals back;
  \item |waitForSignalUp| is used by the parent to wait for the child to be ready;
  \item |signalDown| is used to indicate to the child that the synchronisation has completed.
\end{itemize}

Internally, the |Signal| object makes use of a private Boolean variable |state| with |true| indicating that a child is waiting and |false| otherwise. The use of this variable is protected by a monitor. The Scala code for the |Signal| object can be found in figure \ref{scala:Signal}.
\newpage 
\begin{scala}[label=scala:Signal, caption={The Scala code for the {\scalastyle Signal} object}]
private class Signal{
  /** The state of this object.  true represents that the child has signalled,
    * but not yet received a signal back. */
  private var state = false

  /** Signal to the parent, and wait for a signal back. */
  def signalUpAndWait = synchronized{
    require(!state, 
      "Illegal state of Barrier: this might mean that it is\n"+
      "being used by two threads with the same identity.");
    state = true; notify()
    while(state) wait()
  }

  /** Wait for a signal from the child. */
  def waitForSignalUp = synchronized{ while(!state) wait() }

  /** Signal to the child. */
  def signalDown = synchronized{ state = false; notify() }
}
\end{scala}
%   \caption{The Scala code for the  object}
%   \label{listing:Signal}
% \end{figure}

The |signalUpAndWait| function first asserts that there currently is no other child waiting for the parent to complete a |signalDown|; this ensures that we do not have more than one child using the same |Signal| object. It then sets |state| to |true|, indicating that the child is now waiting and it notifies the parent, awaking them if they are waiting. It then forces the child thread to wait until the parent sets |state| to |false| and notifies the child; the use of the |while| loop here is to guard against spurious wakeups.

|waitForSignalUp| is used by a parent to wait for the child node to perform a |signalUpAndWait| and notify the parent; the |while| loop again guards against spurious wakeups.

The |signalDown| function is used to signal to the child that the synchronisation has been completed and that the child can return from its |signalUpAndWait| call.

\subsubsection{Modelling the Signal objects}

We use our model for a JVM monitor from \framebox{REF} in our modelling of the |Signal| object. 
\framebox{Wording} An interacting thread will run the CSP process of the equivalent Scala function, here referred to as |Func|. In each case these first communicates a |callFunc| before running the |syncFunc| process within a |Synchronized| block as outlined previously in section \framebox{REF} \ref{listing::JVMMonitorInterface}.
%In each case the calling |Proc| will first run the procedure |Func| which communicates the function and the |SignalID| that has been called by a given |ThreadID|. 

\framebox{From signal-scala.csp}
 \begin{cspm}[caption={The state variable(s) and the function call channels}]
channel getState, setState : SignalID . ThreadID . Bool
stateChannels(s) = {|getState.s, setState.s|}

State :: (SignalID) -> Proc
State(s) = Var(false, getState.s, setState.s)

channel callSignalUpAndWait, callWaitForSignalUp, callSignalDown : SignalID . ThreadID
signalChannels = {|callSignalUpAndWait, callWaitForSignalUp, callSignalDown|}
 \end{cspm}

 We first initialise the |state| variable, with the |SignalID| parameter in the channels indicating the |Signal| object the variable corresponds to. We also introduce the function call channels as indicated above.

\begin{cspm}[caption={The CSP model of the {\scalastyle signalUpAndWait} function of the {\scalastyle Signal} object}]
SignalUpAndWait :: (SignalID, ThreadID) -> Proc
SignalUpAndWait(s, t) = 
  callSignalUpAndWait.s.t -> Synchronized(SigM.s, t, syncSignalUpAndWait(s, t))
syncSignalUpAndWait(s, t) = 
  getState.s.t?val -> if val == true then DIV -- Required to be false
                      else setState.s.t.true -> 
                          Notify(SigM.s, t); SignalWaitingForFalse(s, t)


SignalWaitingForFalse :: (SignalID, ThreadID) -> Proc
SignalWaitingForFalse(s, t) = 
  getState.s.t?val -> if val == false then SKIP
                      else Wait(SigM.s, t); SignalWaitingForFalse(s, t)                            
\end{cspm}

This models entering a \inlineScala \CSPM{synchronized} block and checks that \CSPM{state} is not true, diverging if so. This divergence is used to model a failed assertion; the rest of the code is a more direct translation.

\begin{cspm}[caption={The CSP model of the {\scalastyle waitForSignalUp} function of the {\scalastyle Signal} object}]
WaitForSignalUp :: (SignalID, ThreadID) -> Proc  
WaitForSignalUp(s, t) = 
  callWaitForSignalUp.s.t -> Synchronized(SigM.s, t, syncWaitForSignalUp(s, t))        
syncWaitForSignalUp(s, t) = 
  getState.s.t?val -> if val == true then  SKIP 
                      else Wait(SigM.s, t); syncWaitForSignalUp(s, t)
\end{cspm}

Again this is a fairly natural model of the Scala code presented earlier; we communicate that thread |t| has called |waitForSignalUp| on Signal object |s|, enter a synchronized block and then simulate the \inlineScala|while| \inlineCSP loop used to guard against spurious wakeups.

\begin{cspm}[caption={The CSP model of the {\scalastyle signalDown} function of the {\scalastyle Signal} object}]
SignalDown :: (SignalID, ThreadID) -> Proc
SignalDown(s, t) = callSignalDown.s.t -> Synchronized(SigM.s, t, syncSignalDown(s, t))
syncSignalDown(s, t) = setState.s.t.false -> Notify(SigM.s, t); SKIP
\end{cspm}

|SignalDown| is the most simple function of the three to model; it obtains the monitor's lock, sets the |state| variable to false and then notifies the child that the synchronisation has completed.

\begin{cspm}[caption={The initialisation of the {\scalastyle Signal} objects, }]
InitialiseSignal(sig, threads) = 
  runWith(SigM.sig, threads [|stateChannels(sig)|] State(sig)) 
      \ union(stateChannels(sig), events)

allStateChannels(sigs) = {|getState.s, setState.s | s <- sigs|}
States(sigs) = ||| s <- sigs @ State(s)
monitors(sigs) = {SigM.s | s <- sigs}

InitialiseSignals(sigs, threads) = 
  runWithMultiple(monitors(sigs), threads [|allStateChannels(sigs)|] States(sigs))  
      \ union(allStateChannels(sigs), events)
\end{cspm}

We finally define processes |InitialiseSignal| and |InitialiseSignals| to initialise a single signal and multiple signals respectively. This is used to synchronise the threads with the interleaving of the state variables and the monitors and then hiding the internal behaviour of the |Signal| objects. |runWith| and |runWithMultiple| are both used to initialise the monitors for each of the |Signal| objects that are being modelled, ensuring mutual exclusion between threads on each of the |Signal| objects.

% The |Signal| process is simply the process representing the state variable, with channels |getState.s| and |setState.s|

\subsection{The Barrier object}
\inlineScala
When initialised, the |Barrier(n: Int)| object creates an array of |n| |Signal| objects, with these organised in the structure of a heap. As per the trait of a barrier synchronisation, |Barrier| only provides a single function |sync(me)| which takes the thread's identity as an input:

\begin{scala}[caption={The Scala definition of the {\scalastyle Barrier.sync} function}]
/** Perform a barrier synchronisation.
  * @param me the unique identity of this thread. */
def sync(me: Int) = {
  require(0 <= me && me < n, 
    s"Illegal parameter $me for sync: should be in the range [0..$n).")
  val child1 = 2*me+1; val child2 = 2*me+2
  // Wait for children
  if(child1 < n) signals(child1).waitForSignalUp
  if(child2 < n) signals(child2).waitForSignalUp
  // Signal to parent and wait for signal back
  if(me != 0) signals(me).signalUpAndWait
  // Signal to children
  if(child1 < n) signals(child1).signalDown
  if(child2 < n) signals(child2).signalDown
}  
\end{scala}

This checks that the thread's identity is such that |signals(me)| does not cause an 
|ArrayIndexOutOfBoundsException|. It then waits for the thread's children (if they exist) to signal that they are ready to synchronise, before signalling to its parent that all of its children are ready to synchronise. Once the parent signals back that the synchronisation has occurred the thread notfies its children that the synchronisation has completed befoe returning. The exception to this is thread |0|, which has no parent to signal to. Thread |0| reaching line 11 of its |sync(0)| call can therefore be taken as the linearization point of the barrier synchronisation.
%and all of its descendants are ready to synchronise, hence thread |0| reaching line 11 of its execution can be taken as the linearization point of when the barrier synchronisation occurs.

\subsubsection{Modelling the Barrier object}
\inlineCSP
\begin{cspm}[caption={The CSP model of a thread interacting with the {\scalastyle Barrier} object}]
Thread(T.me) = beginSync.T.me -> Sync(T.me) |~| end.T.me -> SKIP

Sync(T.me) = 
  let child1 = 2*me+1 
      child2 = 2*me+2
  within 
      (if (child1 < n) then WaitForSignalUp(S.(child1), T.me) else SKIP);
      (if (child2 < n) then WaitForSignalUp(S.(child2), T.me) else SKIP);
      (if (me != 0) then SignalUpAndWait(S.me, T.me) else SKIP);
      (if (child1 < n) then SignalDown(S.(child1), T.me) else SKIP);
      (if (child2 < n) then SignalDown(S.(child2), T.me) else SKIP);
      endSync.T.me -> Thread(T.me)

Threads = ||| t : ThreadID @ Thread(t)

BarrierSystem = InitialiseSignals(Threads)
\end{cspm}

We recall from earlier that |datatype ThreadID = T.{0..n-1}| and |datatype SignalID = S.{0..n-1}|. The process |Thread(T.me)| models the individual behaviour of a specific thread with identity |T.me :: ThreadID|, with each thread nondeterministically choosing to either communicate an |end.T.me| and terminate or to call the CSP model of |sync(me)|. In the latter case, a communication of |beginSync.T.me| is used to indicate the start of the synchronisation. The |Sync(T.me)| definition is very straightforward, with it mostly following directly from the Scala definition; the only further change is that |Sync(T.me)| communicates a |endSync.T.me| event just before it terminates.

The |thread| processes are then interleaved together to yield |Threads|. We then initialise the system with |Signal| objects that can nondeterministically allow or block spurious wakeups to give |BarrierSystem|. We hide all events of the signal object, so the only visible channels of |BarrierSystem| are |{|beginSync, endSync, spuriousWakeup, end|}|.

\subsection{Correctness of the model}

We will show that the barrier synchronisation is correct; here correct requires that the synchronisation can be correctly linearised and if a synchronisation is possible then it will always occur. 

Correctly linearised in this context means that the barrier synchronisation can be considered to occur at some point between when all |n| threads have communicated |beginSync| and when the first thread communicates an |endSync| event. The requirement that a linearisation must occur means that if all |n| threads communicate a |beginSync| then none of the threads can be blocked from communicating their respective |endSync|.

\begin{cspm}[caption={The lineariser specification for barrier synchronisations}]
Lineariser(t) = beginSync.t -> sync -> endSync.t -> Lineariser(t)
              |~| end.t -> STOP
Spec = ( || t <- ThreadID @ [{beginSync.t, sync, endSync.t, end.t}] 
              Lineariser(t)) \ {sync}
\end{cspm}

|Lineariser(t)| allows any thread to |beginSync.t| followed by an |endSync.t|, representing the call and return of |barrier.sync()|. The |sync| event can be considered to be the point at which the barrier synchronisation occurs since all threads must synchronise on this, fulfilling the requirement above. Additionally, each thread can terminate via |end.t|, indicating that it will perform no further synchronisations. This blocks all other threads from completing a barrier synchronisation, which is the intended behaviour.

We first note that |BarrierSystem| is divergence-free, but |BarrierSystem \ {|spuriousWakeup|}| is not. |BarrierSystem| with spurious wakeups visisble being divergence-free is relevant as this means that we never breach the assertion in the |SignalUpAndWait| function; this therefore means hiding the |spuriousWakeup| events must be the cause of the divergences\framebox{check that isn't an internal divergence caused by hiding spuriousWakeup}. This is expected behaviour as one thread could spuriously wakeup, check the test condition and wait again before spuriously waking up like this indefinitely. Similarly to before this is not a particular cause for concern; in practice spurious wakeups occur infrequently within the JVM.

We first consider the traces model, where we have that the following holds:
\begin{cspm}
assert Spec [T= BarrierSystem \ {|spuriousWakeup|}
\end{cspm}
This means that |BarrierSystem| fulfils the requirements fulfilled by |Spec| i.e.~that it can be linearised and that the synchronisation between all |n| threads occurs correctly (if indeed it does occur).

Since |Spec| cannot diverge we will also consider refinement under the stable failures model This ensures that if a synchronisation can occur then it must occur and all threads communicating |callSync| can return if all |n| threads communicate a |callSync|.
%Using the stable failures model is valid as we only care about the behaviour of the |BarrierSpurSystem| in its stable states (i.e.~where the |Reg| process nondeterministically blocks |spuriousWakeup| events); it could perform an indefinite number of successive |spuriousWakeup| events, but this is an issue with the underlying JVM and not our barrier synchronisation object. 
We note that using the stable failures model is normally inappropriate for a system that can diverge. However, this is valid here as for any state that could be unstable due to a hidden |spuriousWakeup| there exists a corresponding stable state where the regulator process |Reg| blocks the spurious wakeup. FDR yields that both the following hold for systems of upto 6 threads in \framebox{Time value}: 1277 seconds

\begin{cspm}
assert Spec [F= BarrierSystem \ {|spuriousWakeup|}
\end{cspm}

As a result, we have that the |Barrier| object presented earlier is a correct implementation of barrier synchronisation for $n$ upto 6.\framebox{confidence arg here?}

\subsection{Specification processes for the Signal objects}
\framebox{Check}
Our current model of |Signal| models the internal workings of the object, modelling the |synchronized| blocks and the internal |state| variable. Though this is a faithful recreation, this is a rather complex model and leads to significant state space explosion, resulting in us only being able to test for correctness on models with upto 6 threads.
%the number of states that FDR generate being exponential in the number of threads, resulting in us only being able to test refinement on a system of size 6 in reasonable time. 
We can instead construct a specification process which models the use of the |Signal| object. Though this still results in a model size exponential in the number of threads, the model will be of significantly smaller size allowing us to model the |Barrier| object for larger numbers of threads in the same approximate time.

By inspecting the usage of |Signal| we observe that there are two synchronisations between threads performed by each |Signal| object \framebox{Diagram}
\inlineScala
\begin{itemize}
  \item |waitForSignalUp| and |signalUpAndWait| synchronise to indicate that that all threads using objects in this subtree are waiting to synchronise. This synchronisation has the parent waiting on the child to signal, with the child being allowed to signal and progress immediately
  \item |signalDown| and |signalUpAndWait| synchronise, with the parent signalling to the child that the barrier synchronisation has occurred and that |signalUpAndWait| can return. This synchronisation has the child thread waiting on the parent signalling down to it; the child thread is always waiting first as the child starts waiting on this synchronisation immediately after the previous synchronisation occurs.
\end{itemize}
\inlineCSP
We can model this simplified |Signal| object via the following CSP:

  \begin{cspm}[caption={The CSP model of the specification {\scalastyle Signal} object}]
channel endWaitForSignalUp, endSignalUpAndWait : SignalID . ThreadID
waitChannels = {|endWaitForSignalUp, endSignalUpAndWait|}

-- Simplified spec for a correctly used Signal object
SpecSig(s) = 
    callSignalUpAndWait.s?t -> callWaitForSignalUp.s?t2 -> SpecSig2(s, t, t2)
  [] callWaitForSignalUp.s?t2 -> callSignalUpAndWait.s?t -> SpecSig2(s, t, t2)
SpecSig2(s, t, t2) = 
  endWaitForSignalUp.s.t2 -> callSignalDown.s.t2 -> endSignalUpAndWait.s.t -> SpecSig(s)

-- The individual functions for the Signal object
SpecSignalUpAndWait(s, t) = callSignalUpAndWait.s.t -> endSignalUpAndWait.s.t -> SKIP
SpecWaitForSignalUp(s, t) = callWaitForSignalUp.s.t -> endWaitForSignalUp.s.t -> SKIP
SpecSignalDown(s, t) = callSignalDown.s.t -> SKIP

-- Construct the system for each of the SpecSig objects
SpecSignals = 
  || s <- SignalID @ [{|callSignalUpAndWait.s, callWaitForSignalUp.s, 
                        callSignalDown.s, endWaitForSignalUp.s, endSignalUpAndWait.s|}] 
                      SpecSig(s)

-- Method for barrier-sync to initialise the two objects
InitialiseSpecSignals(threads) = 
  (SpecSignals [|union(signalChannels, waitChannels)|] threads) \ waitChannels                     
  \end{cspm} 

We introduce channels |endWaitForSignalUp| and |endSignalUpAndWait| to represent the synchronisations between the child and parent, with each of the channels indicating that their respective functions are able to return. |SpecSig(s)| is used to dictate the order that communications are allowed to occur:

\begin{enumerate}
  \item Initially, it can either communicate a |callSignalUpAndWait| from the child thread or a |callWaitForSignalUp| from the parent. It then communicates the other event.
  \item It then communicates an |endWaitForSignalUp| to indicate to the parent that the first synchronisation has occurred.
  \item The parent then commmunicates a |callSignalDown| indicating that the barrier synchronisation has occured.
  \item Finally, a |endSignalUpAndWait| is communicated to indicate to the child that they can now terminate; |SpecSig(s)| then repeats.
\end{enumerate}

We also define the specification versions of the three external methods offered by a |Signal| object. |SpecSignalUpAndWait| and |SpecWaitForSignalUp| both initially communicate an event indicating that they have been `called' before communicating a |endSignalUpAndWait| or |endWaitForSignalUp| respectively before terminating. By contrast, |SpecSignalDown| immediately terminates after communicating that it has been `called' as it does not require a synchronisation with another thread.

Finally for the |Signal| specifications, we let |SpecSignals| be the alphabetised parallel composition of each of the |SpecSig(s)| processes, with the parallel composition forcing each specification object to only synchronise on events with the matching |SignalID|. The individual threads and the overall system are defined similaly to the above, with the exception that all calls are to the specification processes and not the originals.

\framebox{Stuff about how the initial implementation of Signal fulfils this specification}

\begin{cspm}[caption={The implementation of the {\scalastyle Barrier} based on }]
-- sThread is the same as Thread but uses the spec Signal
sThread(T.me) = beginSync.T.me -> sSync(T.me) |~| end.T.me -> SKIP
sSync(T.me) = 
  let child1 = 2*me+1 
      child2 = 2*me+2
  within (if (child1 < n) then SpecWaitForSignalUp(S.(child1), T.me) else SKIP);
        (if (child2 < n) then SpecWaitForSignalUp(S.(child2), T.me) else SKIP);
        (if me != 0 then SpecSignalUpAndWait(S.me, T.me) else SKIP);
        (if (child1 < n) then SpecSignalDown(S.(child1), T.me) else SKIP);
        (if (child2 < n) then SpecSignalDown(S.(child2), T.me)else SKIP);
        endSync.T.me -> sThread(T.me)

-- Initialise the simple system
sThreads = ||| t : ThreadID @ sThread(t)     
sBarrierSystem = InitialiseSpecSignals(sThreads)

-- Spec failure-divergences refines it as expected
assert Spec [FD= sBarrierSystem
\end{cspm}

Since the simplified |Signal| object does not use monitors we have that the system should be divergence-free. This is verififed by FDR as the above refinement holds against our (divergence-free) linearization checker.


\framebox{Proper performance comparison Simplified can run 10 threads in about the same time}

\framebox{the normal version can run 6}







 
 %\section{Conclusions and future work}

In this paper, we have examined a range of concurrency primitives offered by the Java Virtual Machine, Scala Concurrency Library module and a range of different lock designs. We have examined and proved the correctness of each of these whilst also proving complexity results and examining other properties. There are, however, some limitations to our work. 

Firstly, by the nature of model checking, we are only able to model a limited number of threads with restrictions on other parameters too. Though model checking with larger numbers of threads is technially possible, the exponential blow up in the number of states renders it practically infeasible. If a model is correct for small numbers of threads, we have significantly more confidence in the model remaining correct for larger numbers of threads; we do however note that this does not necessarily imply correctness. 

Take, for example, the SCL monitor which we have previously proved correct for six threads and two conditions in section \ref{section:SCLMonitor}. The components of each individual condition are distinct with the exception of the |ThreadInfo| objects; any flaw with this would be expected to show itself with only two conditions. Likewise any issue due to interfering threads must require at least seven interacting threads; this seems remarkably unlikely due to the simplicity of the system and limited possible interactions between threads. 

We have also only considered a number of specific concurrency primitives. Though our approach can be extended to many other primitives, this would still require significant work to verify the correctness of these. An automated translation system from normal code to CSP would aid in this task. Specialised examples do exist, such as SVA for shared variable programs \cite{RoscoeSVA}, however no mor general models exist. It is worth noting that any general translator would likely suffer from additional complexity blow up due to a lack of insight. An example of this can be seen in the SCL Monitor model, where a more natural implementation of the queue lead to state space- $n!$ times larger than a model with additional insight. Though a na\"{\i}ve translation is very feasible, the utility of such an approach is limited, though non-zero. Implementing an general automated translator, either optimised or na\"{\i}ve, is beyond the scope of the project and therefore left as further work. 

  %\begin{enumerate}
  %  \item \textbf{Action on Objectives:} The attacker now begins to achieve their initial objectives making use of the system access they have achieved. These objectives can include anything from data exfiltration to ransomware attacks to damage to physical systems.
  %\end{enumerate}

  %\begin{itemize}
  %  \item $tree$ indicates the attack tree that the transition and states belong to.
  %\end{itemize}
  
  %\begin{figure}[htp]
  %  \centering
  %  \includegraphics[width=0.7\linewidth]{Attack Tree.jpg}
  %  \caption[Stuxnet installation attack tree]{An attack tree for the installation of Stuxnet}
  %  \label{fig:Stuxnet Installation}
  %\end{figure}

  %\begin{table}
  %  \caption[EternalBlue evaluation results table]{Evaluation results using the EternalBlue attack tree in Figure 6}
  %  \begin{tabularx}{\linewidth}{|l|l|X|X|X|X|l|}
  %    \hline
  %    & Exploit & \multicolumn{4}{l|}{Number of transitions} & \\
  %    Dataset & Present & $|\phi_{0}|$ & $|\phi_{1}|$ & $|\phi_{2}|$  & $|\phi_{3}|$ & Detected \\
  %    \hline
  %    WannaCry Attack\cite{Chen_Wang_Zimba_2019} & Yes & 14 & 220 & 18 & 7742 & Yes \\
  %    EternalBlue.pcp\cite{Ullrich_2017} & Yes & 3 & 48 & 6 & 1 & Yes \\
  %    \hline
  %  \end{tabularx}
  %\end{table}






















  


  \Urlmuskip=0mu plus 1mu
  \emergencystretch=1em
  \printbibliography{}
\end{document}