\inlineScala
\section{Monitors}
\framebox{This is intended to be used in an earlier background section}

A monitor can be used to ensure that certain operations on an object can only be performed under mutual exclusion. Here we first consider the implementation of a monitor used by the Java Virtual Machine (JVM), before considering an alternative implementation that addresses some of the limitations of the JVM monitor.

\subsection{The JVM monitor}

Mutual exclusion between function calls is provided inside the JVM via |synchronized| blocks. Only one thread is allowed to be active inside the synchronized blocks of an object at any point; a separate thread trying to execute a |synchronized| expression will have to wait for the former to release the lock before proceeding. Inside a |synchronized| block, a thread can also call |wait()| to suspend and give up the lock. This waits until a separate thread (which can now proceed) executes a |notify()|, which will wake the waiting thread and allow it to proceed once the notifying thread has released the lock.

It is important to note that the implementation of |wait()| is buggy. Sometimes a thread that has called |wait()| will wake up even without a |notify()|; this is called a \emph{spurious wakeup}.




\subsubsection{Modelling the JVM monitor}

\inlineCSP

For our model of a monitor in CSP we have extended the |JVMMonitor| provided by Lowe \cite{LoweJVMMonitor}.

Lowe's module previously provided a single monitor; this is problematic in case we have multiple objects which each require their own monitor. We instead introduce a datatype |MonitorID|, with this type containing all possible 'objects' that could require their own monitors. The |JVMMonitor| module is then changed to be parameterised over some subset of |MonitorID|. The internal channels and processes now also take some |MonitorID| value to identify which object is being reffered to at any point.

%This required changing the module to parameterise it over the |Monitor| type, with this containing all possible 'objects' that could require a monitor, allowing for each object we model to have its own, independent monitor; this is similar to how the JVM assigns each object its own monitor. The original implementation requires that a new |JVMMonitor| be instantiated every time a unique monitor was required; this is very fiddly for when we have a variable number of monitors required. Instead the internal channels are now of the type \inlineCSP |Monitor . ThreadID|, allowing for easy identification of which object the monitor accompanies.

Internally, the model uses one lock per |MonitorID|. These also have an additional parameter storing the identities of any waiting threads\framebox{full code in an appendix?}. So that it is a faithful model of an actual JVMMonitor we also model spurious wakeups via the |spuriousWakeup| channel. We therefore run the the lock process in parallel with a regulator process |Reg = CHAOS({|spuriousWakeup|})|; this is used to non-deterministically allow or block spurious wakeups where appropriate. This is important as when we hide |spuriousWakeup| events we have that every state in FDR that allows a |spuriousWakeup| has a pair which blocks the |spuriousWakeup|. This allows us to perform refinement checks in the stable-failures model instead of just the failures-divergences model, allowing us to write more natural specification processes.

Our model of the monitor provides the following exports:

\begin{figure}[H]
\begin{cspm}
module JVMMonitor"(MonitorID)"
    ...
    Reg = CHAOS({|spuriousWakeup|})
exports

    -- All events except spuriousWakeup
    events = {| acquire, release, wait, notify, notifyAll |}

    channel spuriousWakeup : "MonitorID" . ThreadID
  
    "InitialiseAll ="
      "||| mon <- MonitorID @ (Unlocked(mon, {})  [| {|spuriousWakeup|} |] Reg)"

    runWith("obj", P) = P [| events |] (Unlocked("obj", {})  [| {|spuriousWakeup|} |] Reg)

    -- Interface to threads.

    -- Lock the monitor
    Lock("obj", t) = ...

    -- Unlock the monitor
    Unlock("obj", t) = ...

    -- Perform P under mutual exclusion
    Synchronized("obj", t, P) = ...

    -- Perform P under mutual exclusion, and apply cont to the result. 
    -- MutexC :: (ThreadID, ((a) -> Proc) -> Proc, (a) -> Proc) -> Proc
    SynchronizedC("obj", t, P, cont) = ...

    -- Perform a wait(), and then regain the lock.
    Wait("obj", t) = ... 

    -- perform a notify()
    Notify("obj", t) = ...

    -- perform a notifyAll()
    NotifyAll("obj", t) = ...

endmodule
\end{cspm}
\caption{The interface of the JVMMonitor module; changes are underlined}
\end{figure}

Each monitor provides |Wait(o, t)|, |Notify(o, t)| and |Synchronized(o, t, Proc)| methods to model the equivalent functions/blocks in SCL. The |Proc| parameter is used to specify the process that will be run inside the |Synchronized| block; the intended usage of this is of the form |callFunc.o.t -> Synchronized(o, t, syncFunc)| where the process communicates that it is calling the model of fucntion |Func| before completing the rest of the function whilst holding the monitor lock.

% \subsection{|LockSupport|}

% \lstinputlisting{C:\Users\tom\Documents\repos\4yp\new-csp\lock-support-module.csp}

